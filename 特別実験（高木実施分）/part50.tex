\section{角度制御}
\subsection{P制御}
\subsubsection{パラメータ設定}

4章で述べたように,P コントローラ
\begin{align}
    v_x(t) &= k_{Px} e_x(t),\quad e_x(t) = \theta_x^{\mathrm{ref}}(t) - \theta_x(t) \notag \\
    v_y(t) &= k_{Py} e_y(t),\quad e_y(t) = \theta_y^{\mathrm{ref}}(t) - \theta_y(t)
    \tag{5.1}
\end{align}
を用いると,$\theta_x^{\mathrm{ref}}(s)$ から $\theta_x(s)$ への伝達関数は 2 次遅れ要素
\begin{align}
    \begin{cases}
        T_x(s) = \dfrac{\omega_{nx}^2}{s^2 + 2\zeta_x \omega_{nx} s + \omega_{nx}^2} \\
        T_y(s) = \dfrac{\omega_{ny}^2}{s^2 + 2\zeta_y \omega_{ny} s + \omega_{ny}^2}
    \end{cases}
    \tag{5.2}
\end{align}
となる.ただし,
\begin{tcolorbox}[colframe=black!75!white, colback=white!95!white, boxrule=0.5pt]
\textbf{固有角周波数:} 
$\omega_{nx} = \sqrt{b_x k_{Px}},\quad \omega_{ny} = \sqrt{b_y k_{Py}}$ \\
\textbf{減衰係数:} 
$\zeta_x = \dfrac{a_x}{2\omega_{nx}} = \dfrac{a_x}{2\sqrt{b_x k_{Px}}},\quad
\zeta_y = \dfrac{a_y}{2\omega_{ny}} = \dfrac{a_y}{2\sqrt{b_y k_{Py}}}$
\end{tcolorbox}
である.したがって,P コントローラで指定できるのは,固有角周波数 $\omega_{nx}$,$\omega_{ny}$ か減衰係数 $\zeta_x$,$\zeta_y$ のいずれかである.

たとえば,固有角周波数 $\omega_{nx}$,$\omega_{ny}$ を指定した値 $\omega_{Mx}$,$\omega_{My}$ とするとには,比例ゲイン $k_{Px}$,$k_{Py}$ を
\begin{align}
    \begin{cases}
        k_{Px} = \dfrac{\omega_{Mx}^2}{b_x} \\
        k_{Py} = \dfrac{\omega_{My}^2}{b_y}
    \end{cases}
    \tag{5.4}
\end{align}
とすればよいが,$\omega_{Mx}, \omega_{My}$ を大きくするにつれて比例ゲイン $k_{Px}$,$k_{Py}$ が大きくなり,減衰係数 $\zeta_x$, $\zeta_y$ を零に近づけてしまうため,振動的な応答となってしまう.

また,減衰係数 $\zeta_x$, $\zeta_y$ を指定した値(たとえば $\zeta_{Mx}, \zeta_{My}$)とするとには,比例ゲイン $k_{Px}$,$k_{Py}$ を
\begin{align}
    \begin{cases}
        k_{Px} = \dfrac{a_x^2}{4\zeta_{Mx}^2 b_x} \\
        k_{Py} = \dfrac{a_y^2}{4\zeta_{My}^2 b_y}
    \end{cases}
    \tag{5.5}
\end{align}
とすればよいが,固有角周波数 $\omega_{nx}, \omega_{ny}$ が決まってしまい,速応性を考慮することはできない.

\subsubsection{シミュレーションと実験}

\noindent
\fbox{\strut \textbf{ステップ 1}}:
MATLAB/Simulink を起動し,カレントディレクトリを \texttt{\textyen Pcont} に移動する.つぎに,4.3 節で定めたパラメータを保存した MAT ファイル ``\texttt{armpara.mat}'',(5.4) 式にしたがって P コントローラを設計する M ファイル ``\texttt{armP.m}'',シミュレーションモデル ``\texttt{sim\_P.slx}''(図\ref{fig:simulink_model_4link2joint}),実機実験モデル ``\texttt{ex\_P.slx}''(図\ref{fig:sim_param})を作成し,\texttt{C:\textbackslash robot\textbackslash Pcont} に保存する.ただし,目標値は
\begin{align}
    \mathrm{Step}\left(\theta_x^{\mathrm{ref}}(t)\right)\colon
    \begin{cases}
        \text{ステップ時間:}0 \\
        \text{初期値:}0 \\
        \text{最終値:}0.3 \\
        \text{サンプル時間:}0
    \end{cases}
    ,\quad
    \mathrm{Step1}\left(\theta_y^{\mathrm{ref}}(t)\right)\colon
    \begin{cases}
        \text{ステップ時間:}0 \\
        \text{初期値:}0 \\
        \text{最終値:}0 \\
        \text{サンプル時間:}0
    \end{cases}
    \tag{5.6}
\end{align}
と設定する.

\begin{figure}[htbp]
    \centering
    \begin{subfigure}[b]{0.45\linewidth}
        \centering
        \includegraphics[width=\linewidth]{figure/sim_result_omega.png}
        \caption{シミュレーションモデル ``sim\_P.slx''}
    \end{subfigure}
    \hfill
    \begin{subfigure}[b]{0.45\linewidth}
        \centering
        \includegraphics[width=\linewidth]{figure/sim_model_block.png}
        \caption{Subsystem ``2D Robot Model'' の内容}
    \end{subfigure}
    \caption{P 制御の Simulink モデル}
    \label{fig:sim_param}
\end{figure}

\noindent
モデルウィンドウのメニューで ``シミュレーション/モデル コンフィグ...'' を選択し,
\begin{itemize}
  \item 開始時間:0
  \item 終了時間:1
  \item ソルバー:ode4 (Runge-Kutta)
  \item 固定ステップ:0.01
\end{itemize}
に変更.ブロック ``To Workspace (2,3,4,5,6)'' をダブルクリックし,保存形式を \textbf{配列} に変更する.

\vspace{1em}

\noindent
\fbox{\textbf{ステップ 2}}: ``\texttt{armP\_sim.m}'' を実行し,図\ref{fig:sim_param}~(a) のシミュレーション結果を取得する.

\vspace{1em}
\noindent
\fbox{\textbf{ステップ 3}}:
$\omega_{Mx} = 15$,$\omega_{My} = 15$ としたときの P コントローラを設計するため,M ファイル ``\texttt{armP.m}'' を実行する.その結果,以下の実行結果が得られる.
\newpage

{armP.m} の実行結果
    \begin{verbatim}
    >> armP
    各パラメータを設定して下さい
    omegaMx = 15
    omegaMy = 15
    kPx = 
        2.9302
    kPy = 
        2.9693
    \end{verbatim}
\noindent
つぎに,設計された P コントローラを用いて,``\texttt{ex\_p.slx}'' を実行し,角度制御の実機実験を行う.得られた MAT ファイル ``\texttt{thetax.mat}'' の名前を ``\texttt{thetax15.mat}'' に変更する.

\noindent
\fbox{\textbf{ステップ 4}}:アームの角度を 0 にし,ステップ~3 において $\omega_{Mx} = 20$,$\omega_{My} = 20$ と指定し,同様の作業を行う.ただし,得られた MAT ファイル “\texttt{thetax.mat}” の名前を “\texttt{thetax20.mat}” に変更する.

\noindent
\fbox{\textbf{ステップ 5}}:
アームの角度を 0 にし,ステップ~3 において $\omega_{Mx} = 30$,$\omega_{My} = 30$ と指定し,同様の作業を行う.
ただし,得られた MAT ファイル “\texttt{thetax.mat}” の名前を “\texttt{thetax30.mat}” に変更する.

\noindent
\fbox{\textbf{ステップ 6}}:
MATLAB Command Window で M ファイル ``\texttt{plot\_figure.m}'' を実行する.

\subsubsection{実験結果及び考察}
\begin{figure}[H]
    \centering
    \begin{subfigure}[b]{0.45\linewidth}
        \centering
        \includegraphics[width=\linewidth]{figure/sim_P-crop.pdf}
        \caption{シミュレーション結果}
    \end{subfigure}
    \hfill
    \begin{subfigure}[b]{0.45\linewidth}
        \centering
        \includegraphics[width=\linewidth]{figure/plot_figure-crop.pdf}
        \caption{実験結果}
    \end{subfigure}
    \caption{P 制御のステップ応答}
    \label{fig:step_response}
\end{figure}

\noindent
\textbf{図~\ref{fig:step_response}} より $\omega_{Mx}$ を大きくするにしたがって速応性は改善されていることがわかるが,
オーバーシュートが大きくなっていることがわかる.また,考慮しなかったクローン摩擦の影響によって,
実機実験においては定常偏差が生じていることがわかる.

\subsection{P--D 制御(微分先行型 PD 制御)}
\subsubsection{パラメータ設定}

P--D コントローラ(微分先行型 PD コントローラ)

\begin{equation}
\left\{
\begin{aligned}
    v_x(t) &= k_{Px} e_x(t) - k_{Dx} \frac{d\theta_x(t)}{dt} \\
    v_y(t) &= k_{Py} e_y(t) - k_{Dy} \frac{d\theta_y(t)}{dt}
\end{aligned}
\right\}
\quad \Longleftrightarrow \quad
\left\{
\begin{aligned}
    v_x(s) &= k_{Px} e_x(s) - k_{Dx} s \theta_x(s) \\
    v_y(s) &= k_{Py} e_y(s) - k_{Dy} s \theta_y(s)
\end{aligned}
\right.
\tag{5.7}
\end{equation}

を用いると,$\theta_x^{\text{ref}}(s)$ から $\theta_x(s)$ への伝達関数は 2 次遅れ要素

\begin{equation}
\left\{
\begin{aligned}
    T_x(s) &= \frac{\omega_{nx}^2}{s^2 + 2\zeta_x \omega_{nx}s + \omega_{nx}^2} \\
    T_y(s) &= \frac{\omega_{ny}^2}{s^2 + 2\zeta_y \omega_{ny}s + \omega_{ny}^2}
\end{aligned}
\right.
\tag{5.8}
\end{equation}

となる.ただし,

\begin{equation}
\left\{
\begin{aligned}
    \text{固有角周波数:} &\quad \omega_{nx} = \sqrt{b_x k_{Px}}, \quad \omega_{ny} = \sqrt{b_y k_{Py}} \\
    \text{減衰係数:} &\quad \zeta_x = \frac{a_x + b_x k_{Dx}}{2 \omega_{nx}}, \quad \zeta_y = \frac{a_y + b_y k_{Dy}}{2 \omega_{ny}}
\end{aligned}
\right.
\tag{5.9}
\end{equation}

である.したがって,P--D 制御では比例ゲイン $k_{Px}, k_{Py}$ により速応性に関するパラメータ $\omega_{nx}, \omega_{ny}$ を指定し,
微分ゲイン $k_{Dx}, k_{Dy}$ により減衰性に関するパラメータ $\zeta_x, \zeta_y$ を指定することができる.つまり,固有角周波数
$\omega_{nx}, \omega_{ny}$ および減衰係数 $\zeta_x, \zeta_y$ を指定した値 $\omega_{Mx}, \omega_{My}, \zeta_{Mx}, \zeta_{My}$ とするには,

\begin{equation}
\left\{
\begin{aligned}
    k_{Px} &= \frac{\omega_{Mx}^2}{b_x}, \quad
    k_{Dx} = \frac{2\zeta_{Mx} \omega_{Mx} - a_x}{b_x} \\
    k_{Py} &= \frac{\omega_{My}^2}{b_y}, \quad
    k_{Dy} = \frac{2\zeta_{My} \omega_{My} - a_y}{b_y}
\end{aligned}
\right.
\tag{5.10}
\end{equation}

なお,ポテンショメータによって検出された角度には高周波成分の観測雑音(ノイズ)が含まれているため,検出された角度をもとに角速度を算出すると,インパルス状の成分を含んでしまう.
そこで,実際には,検出された角度を 1 次のローパスフィルタ
\begin{equation}
    G_{fx}(s) = \frac{1}{1 + T_{dx}s}, \quad G_{fy}(s) = \frac{1}{1 + T_{dy}s}
\end{equation}
に通して高周波成分の観測雑音を除去した後,角度信号を微分する必要がある.以上のことを考慮すると,P--Dコントローラは次式のようになる.

\begin{equation}
\left\{
\begin{aligned}
    v_x(s) &= k_{Px} e_x(s) - \frac{k_{Dx}s}{1 + T_{dx}s} \theta_x(s) \\
    v_y(s) &= k_{Py} e_y(s) - \frac{k_{Dy}s}{1 + T_{dy}s} \theta_y(s)
\end{aligned}
\right.
\tag{5.11}
\end{equation}

\subsubsection{シミュレーションと実験}
\fbox{\textbf{ステップ 1}}:
パラメータを記述した M ファイル ``armpara.m'' および (5.11) 式にしたがって P--D コントローラのパラメータを設計する M ファイル ``armPD.m'',シミュレーションモデル ``sim\_PD.slx''(図 5.3 (a)),実機実験モデル ``ex\_PD.slx''(図 5.3 (b))を作成し,C:\textbackslash robot\textbackslash PDcont に保存する.

\begin{figure}[H]
    \centering
    \begin{subfigure}{0.8\linewidth}
        \centering
        \includegraphics[width=\linewidth]{figure/sim_PD_slx.pdf}
        \caption{シミュレーションモデル ``sim\_PD.slx''}
    \end{subfigure}
    
    \vspace{0.5cm}
    
    \begin{subfigure}{0.8\linewidth}
        \centering
        \includegraphics[width=\linewidth]{figure/ex_PD_slx.pdf}
        \caption{実機実験モデル ``ex\_PD.slx''}
    \end{subfigure}
    
    \vspace{0.5cm}
    
    \begin{subfigure}{0.8\linewidth}
        \centering
        \includegraphics[width=\linewidth]{figure/PD_Controller_detail.pdf}
        \caption{Subsystem ``PD Controller'', ``PD Controller1'' の内容}
    \end{subfigure}
    
    \caption{P--D 制御の Simulink モデル}
    \label{fig:pd_simulink_model}
\end{figure}

\noindent
\fbox{\textbf{ステップ 2}}:
アームの角度を 0 にし,$\omega_{Mx} = 30$, $\zeta_{Mx} = 0.7$ と指定し,実験する.ただし,得られた MAT ファイル ``\texttt{thetax.mat}'' の名前を ``\texttt{thetax7.mat}'' に変更する.

\vspace{1em}

\noindent
\fbox{\textbf{ステップ 3}}:
アームの角度を 0 にし,$\omega_{Mx} = 30$, $\zeta_{Mx} = 1.0$ と指定し,実験する.ただし,得られた MAT ファイル ``\texttt{thetax.mat}'' の名前を ``\texttt{thetax1.mat}'' に変更する.

\subsubsection{実験結果及び考察}

シミュレーションおよび実機実験を行うと,\textbf{図 5.4} (a) のシミュレーション結果および \textbf{図 5.4} (b) の実験結果が得られる.P 制御と比べて安定度が改善されていることが確認できる.しかしながら,実機実験では,P 制御と同様,クローン摩擦などの影響で定常偏差が残っている.

\begin{figure}[H]
    \centering
    \begin{minipage}{0.48\linewidth}
        \centering
        \includegraphics[width=\linewidth]{figure/ex_pd_sim-crop.pdf}
        \subcaption{シミュレーション結果}
    \end{minipage}
    \hfill
    \begin{minipage}{0.48\linewidth}
        \centering
        \includegraphics[width=\linewidth]{figure/ex_pd_hidari-crop.pdf}
        \subcaption{実験結果}
    \end{minipage}
    \caption{P--D 制御のステップ応答}
    \label{fig:pd_response}
\end{figure}

\subsubsection{パラメータ設定}

P 制御や P--D 制御はコントローラに積分器 \(1/s\) を含んでいないため,クローン摩擦の影響でステップ応答に定常偏差が生じた.そこで,コントローラに積分器 \(1/s\) を含ませることによって,ステップ状の目標値に対する定常偏差を解消することを考える.

I--PD コントローラ(比例・微分先行型 PID コントローラ)

\begin{equation}
\left\{
\begin{aligned}
v_x(t) &= -k_{Px} \theta_x(t) + k_{Ix} \int_0^t e_x(\tau) d\tau - k_{Dx} \frac{d\theta_x(t)}{dt} \\
v_y(t) &= -k_{Py} \theta_y(t) + k_{Iy} \int_0^t e_y(\tau) d\tau - k_{Dy} \frac{d\theta_y(t)}{dt}
\end{aligned}
\right.
\Longleftrightarrow
\left\{
\begin{aligned}
v_x(s) &= -k_{Px} \theta_x(s) + \frac{k_{Ix}}{s} e_x(s) - k_{Dx}s \theta_x(s) \\
v_y(s) &= -k_{Py} \theta_y(s) + \frac{k_{Iy}}{s} e_y(s) - k_{Dy}s \theta_y(s)
\end{aligned}
\right.
\tag{5.12}
\end{equation}

を用いると,\( \theta_x^{ref}(s) \) から \( \theta_x(s) \) への伝達関数は 3 次遅れ要素

\begin{equation}
\left\{
\begin{aligned}
T_x(s) &= \frac{b_x k_{Ix}}{s^3 + (a_x + b_x k_{Dx})s^2 + b_x k_{Px}s + b_x k_{Ix}} \\
T_y(s) &= \frac{b_y k_{Iy}}{s^3 + (a_y + b_y k_{Dy})s^2 + b_y k_{Py}s + b_y k_{Iy}}
\end{aligned}
\right.
\tag{5.13}
\end{equation}

となる.したがって,(5.13) 式を規範モデル

\begin{equation}
\left\{
\begin{aligned}
T_{Mx}(s) &= \frac{\omega_{Mx}^3}{s^3 + \alpha_{Mx2} \omega_{Mx}^2 s^2 + \alpha_{Mx1} \omega_{Mx}s + \omega_{Mx}^3} \\
T_{My}(s) &= \frac{\omega_{My}^3}{s^3 + \alpha_{My2} \omega_{My}^2 s^2 + \alpha_{My1} \omega_{My}s + \omega_{My}^3}
\end{aligned}
\right.
\tag{5.14}
\end{equation}

と完全に一致させるには

\begin{equation}
\left\{
\begin{aligned}
k_{Ix} &= \frac{\omega_{Mx}^3}{b_x}, \quad k_{Px} = \frac{\alpha_{Mx1} \omega_{Mx}^2}{b_x}, \quad k_{Dx} = \frac{\alpha_{Mx2} \omega_{Mx} - a_x}{b_x} \\
k_{Iy} &= \frac{\omega_{My}^3}{b_y}, \quad k_{Py} = \frac{\alpha_{My1} \omega_{My}^2}{b_y}, \quad k_{Dy} = \frac{\alpha_{My2} \omega_{My} - a_y}{b_y}
\end{aligned}
\right.
\tag{5.15}
\end{equation}

と選べばよい.ただし,\(\omega_{Mx}, \omega_{My}\) は速度応答に関するパラメータ,\(\alpha_{M1x}, \alpha_{M2x}, \alpha_{M1y}, \alpha_{M2y}\) は減衰性に関するパラメータであり,

\paragraph{規範モデルの標準形}
\begin{itemize}
    \item \textbf{パターワース標準形}:
    \[
    \left\{
    \begin{aligned}
        \alpha_{M1x} = 2, \quad \alpha_{M2x} = 2 \\
        \alpha_{M1y} = 2, \quad \alpha_{M2y} = 2
    \end{aligned}
    \right.
    \]
    
    \item \textbf{二項標準形}:
    \[
    \left\{
    \begin{aligned}
        \alpha_{M1x} = 3, \quad \alpha_{M2x} = 3 \\
        \alpha_{M1y} = 3, \quad \alpha_{M2y} = 3
    \end{aligned}
    \right.
    \]
    
    \item \textbf{ITAE 最小標準形}:
    \[
    \left\{
    \begin{aligned}
        &\alpha_{M1x} = 2.15, &\alpha_{M2x} = 1.75\\
        &\alpha_{M1y} = 2.15, &\alpha_{M2y} = 1.75
    \end{aligned}
    \right.
    \]
\end{itemize}

\noindent
が用いられることが多い.

\bigskip

\noindent
なお,実際には高周波成分の観測雑音を除去するため,次式の I--PD コントローラを用いることになる:

\begin{equation}
\left\{
\begin{aligned}
v_x(s) &= -k_{Px} \theta_x(s) + \frac{k_{Ix}}{s} e_x(s) - \frac{k_{Dx}s}{1 + T_{dx}s} \theta_x(s) \\
v_y(s) &= -k_{Py} \theta_y(s) + \frac{k_{Iy}}{s} e_y(s) - \frac{k_{Dy}s}{1 + T_{dy}s} \theta_y(s)
\end{aligned}
\right.
\tag{5.16}
\end{equation}

\subsubsection{シミュレーションと実験}
\fbox{\textbf{ステップ 1}}:
パラメータを記述した M ファイル ``armpara.m'' および (5.15) 式により I--PD コントローラのパラメータを設計する M ファイル ``armIPD.m'',シミュレーションモデル ``sim\_IPD.slx''(図 5.5 (a)),実機実験モデル ``ex\_IPD.slx''(図 5.5 (b))を作成し,C:\textbackslash robot\textbackslash IPDcont に保存する.

\fbox{\textbf{ステップ 2}}:
アームの角度を 0 にし,$\omega_{Mx} = 30$, $\alpha_{M1x} = 2$, $\alpha_{M2x} = 2$ と指定し,実験する.ただし,得られた MAT ファイル ``thetax.mat'' の名前を ``thetaxb.mat'' に変更する.

\fbox{\textbf{ステップ 3}}:
アームの角度を 0 にし,$\omega_{Mx} = 30$, $\alpha_{M1x} = 3$, $\alpha_{M2x} = 3$ と指定し,実験する.ただし,得られた MAT ファイル ``thetax.mat'' の名前を ``thetax2.mat'' に変更する.

\fbox{\textbf{ステップ 4}}:
アームの角度を 0 にし,$\omega_{Mx} = 30$, $\alpha_{M1x} = 2.15$, $\alpha_{M2x} = 1.75$ と指定し,実験する.ただし,得られた MAT ファイル ``thetax.mat'' の名前を ``thetaxi.mat'' に変更する.

\begin{figure}[H]
    \centering
    \includegraphics[width=0.95\linewidth]{figure/ipd_simulink_model.pdf}
    \caption{I--PD 制御の Simulink モデル}
    \label{fig:ipd_simulink}
\end{figure}

M ファイル ``armIPD.m'' を以下のように実行し,シミュレーションおよび実機実験を行うと,
図~\ref{fig:ipd_simulink}~(a) のシミュレーション結果および
図~\ref{fig:ipd_simulink}~(b) の実験結果が得られる.
図~\ref{fig:ipd_simulink}~(b) より I--PD 制御では定常偏差が零になっていることがわかる.

\subsubsection{実験結果及び考察}
\begin{figure}[H]
    \centering
    \begin{minipage}[b]{0.48\linewidth}
      \centering
      \includegraphics[width=\linewidth]{figure/ex_ipd_sim-crop.pdf}
      \subcaption{シミュレーション結果}
    \end{minipage}
    \begin{minipage}[b]{0.48\linewidth}
      \centering
      \includegraphics[width=\linewidth]{figure/ex_ipd-crop.pdf}
      \subcaption{実験結果}
    \end{minipage}
    \caption{I--PD 制御のステップ応答}
    \label{fig:ipd_result}
  \end{figure}
  
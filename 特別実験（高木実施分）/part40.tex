\section{時間応答に基づくパラメータ同定}
\subsection{パラメータ同定とは?}

制御対象の物理パラメータの値は,はかりやメジャーなどの測定器で直接はかることができるとは限らない.
制御対象の物理パラメータの値がわからないと,制御対象の数学モデルに基づいたコントローラ設計やシミュレーションを行うことができない.
そこで,測定器で直接はかることができない物理パラメータを定めるため,制御対象に様々な信号を加え,このときの出力応答から物理パラメータの値を定める必要がある.このことを\textbf{パラメータ同定}と呼ぶ.

本実験装置の場合,\ref{sec:2_1} 節で述べたように,$v_x(t)$,$v_y(t)$ から $\theta_x(t)$,$\theta_y(t)$ への伝達関数 $G_x(s)$,$G_y(s)$ は
\begin{equation}
\left\{
\begin{aligned}
G_x(s) &= \frac{b_x}{s(s + a_x)}, \quad a_x = \frac{c_x}{J_x}, \quad b_x = \frac{k_{fx}}{J_x}, \\
G_y(s) &= \frac{b_y}{s(s + a_y)}, \quad a_y = \frac{c_y}{J_y}, \quad b_y = \frac{k_{fy}}{J_y}
\end{aligned}
\right.
\label{eq:gx_gy_transfer}
\end{equation}

であるから,$a_x$,$b_x$,$a_y$,$b_y$ の値を実験により定めることになる.
パラメータ同定を行う方法には,ステップ応答などの時間応答に基づく方法と周波数応答に基づく方法があるが,ここでは,時間応答に基づく方法を用いる.

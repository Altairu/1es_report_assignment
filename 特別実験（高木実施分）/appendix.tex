% !TEX root = main.tex
%////////////////////////////////////////////////////////
\begin{center}
    \section*{\kintou{2.5zw}{付録}}                      %% ここに番号をつけない
    \vspace*{-2zh}
\end{center}
\addcontentsline{toc}{section}{付録} %% 目次に番号をつけない
\appendix

\setcounter{equation}{0}
\setcounter{figure}{0}
\setcounter{table}{0}

\makeatletter
     \renewcommand{\theequation}{%
          A.\arabic{equation}}
     \@addtoreset{equation}{section}
\makeatother

\makeatletter
     \renewcommand{\thetable}{%
          A.\arabic{table}}
     \@addtoreset{table}{section}
\makeatother

\makeatletter
     \renewcommand{\thefigure}{%
          A.\arabic{figure}}
     \@addtoreset{figure}{section}
\makeatother

% ================================================
\makeatletter
     \renewcommand{\thesubsection}{%
          A.\arabic{subsection}}
     \@addtoreset{subsection}{section}
\makeatother

\makeatletter
     \renewcommand{\thesubsubsection}{%
          A.\arabic{subsection}.\arabic{subsubsection}}
     \@addtoreset{subsubsection}{section}
\makeatother

% ================================================
\makeatletter
     \renewcommand{\thetheorem}{%
          A.\arabic{theorem}}
     \@addtoreset{theorem}{section}
\makeatother

\makeatletter
     \renewcommand{\thedefinition}{%
          A.\arabic{definition}}
     \@addtoreset{definition}{section}
\makeatother

\makeatletter
     \renewcommand{\thelemma}{%
          A.\arabic{lemma}}
     \@addtoreset{lemma}{section}
\makeatother

\makeatletter
     \renewcommand{\thelstlisting}{%
          {\bf{A.\arabic{lstlisting}}}}
     \@addtoreset{lstlisting}{section}
\makeatother






%////////////////////////////////////////////////////////

%%%%%%%%%%%%%%%%%%%%%%%%%%%%%%%%%%%%%%%%%%%%%%%%%%%%%%
\subsection{下位レイヤープログラム}
%%%%%%%%%%%%%%%%%%%%%%%%%%%%%%%%%%%%%%%%%%%%%%%%%%%%%%

作成したプログラムを以下に示す.
また,\url{https://github.com/Altairu/2_wheel_PID_control-_CubeIED}に掲載している.

\subsubsection{メインプログラム}

\begin{lstlisting}[language=C, caption=メインコード(main.c)]
    #include "main.h"
    #include "motor_driver.h"
    #include "encoder.h"
    #include "serial_lib.h"
    
    TIM_HandleTypeDef htim1;
    TIM_HandleTypeDef htim12;
    TIM_HandleTypeDef htim4;
    TIM_HandleTypeDef htim8;
    UART_HandleTypeDef huart2;
    
    MotorDriver motorRight, motorLeft;
    Encoder encoderRight, encoderLeft;
    EncoderData encoderDataRight, encoderDataLeft;
    
    #define WHEEL_DIAMETER_MM 365 //129.5
    #define ENCODER_PULSES_PER_REV 4096
    
    int16_t controlSignalRight = 0;
    int16_t controlSignalLeft = 0;
    
    void SystemClock_Config(void);
    static void MX_GPIO_Init(void);
    static void MX_USART2_UART_Init(void);
    static void MX_TIM1_Init(void);
    static void MX_TIM12_Init(void);
    static void MX_TIM4_Init(void);
    static void MX_TIM8_Init(void);
    
    int main(void)
    {
        HAL_Init();
        SystemClock_Config();
        MX_GPIO_Init();
        MX_USART2_UART_Init();
        MX_TIM1_Init();
        MX_TIM12_Init();
        MX_TIM4_Init();
        MX_TIM8_Init();
    
        Serial_Init(&huart2);
        MotorDriver_Init(&motorRight, &htim12, TIM_CHANNEL_1, &htim12, TIM_CHANNEL_2);
        MotorDriver_Init(&motorLeft, &htim1, TIM_CHANNEL_4, &htim1, TIM_CHANNEL_1);
        Encoder_Init(&encoderRight, &htim8, WHEEL_DIAMETER_MM, ENCODER_PULSES_PER_REV, 10);
        Encoder_Init(&encoderLeft, &htim4, WHEEL_DIAMETER_MM, ENCODER_PULSES_PER_REV, 10);
    
        uint32_t lastSendTime = HAL_GetTick();
    
        while (1)
        {
            /* PCからの制御信号を受信 */
            int16_t receivedData[2];
            if (Serial_ReceiveData(&huart2, receivedData, 2))
            {
                controlSignalRight = receivedData[0];
                controlSignalLeft = receivedData[1];
    
                MotorDriver_setSpeed(&motorRight, -1* controlSignalRight);
                MotorDriver_setSpeed(&motorLeft,   controlSignalLeft);
            }
    
            /* エンコーダーデータの送信(10msごとに送信) */
            if (HAL_GetTick() - lastSendTime >= 10)
            {
                lastSendTime = HAL_GetTick();
    
                /* エンコーダーの速度データを更新 */
                Encoder_Interrupt(&encoderRight, &encoderDataRight);
                Encoder_Interrupt(&encoderLeft, &encoderDataLeft);
    
                /* エンコーダー速度を送信 */
                int16_t feedbackData[2] = {(int16_t)encoderDataRight.velocity, -1*(int16_t)encoderDataLeft.velocity};
                Serial_SendData(&huart2, feedbackData, 2);
            }
        }
    }
    
    
    /**
      * @brief System Clock Configuration
      * @retval None
      */
    void SystemClock_Config(void)
    {
      RCC_OscInitTypeDef RCC_OscInitStruct = {0};
      RCC_ClkInitTypeDef RCC_ClkInitStruct = {0};
    
      /** Configure the main internal regulator output voltage
      */
      __HAL_RCC_PWR_CLK_ENABLE();
      __HAL_PWR_VOLTAGESCALING_CONFIG(PWR_REGULATOR_VOLTAGE_SCALE3);
    
      /** Initializes the RCC Oscillators according to the specified parameters
      * in the RCC_OscInitTypeDef structure.
      */
      RCC_OscInitStruct.OscillatorType = RCC_OSCILLATORTYPE_HSI;
      RCC_OscInitStruct.HSIState = RCC_HSI_ON;
      RCC_OscInitStruct.HSICalibrationValue = RCC_HSICALIBRATION_DEFAULT;
      RCC_OscInitStruct.PLL.PLLState = RCC_PLL_ON;
      RCC_OscInitStruct.PLL.PLLSource = RCC_PLLSOURCE_HSI;
      RCC_OscInitStruct.PLL.PLLM = 16;
      RCC_OscInitStruct.PLL.PLLN = 336;
      RCC_OscInitStruct.PLL.PLLP = RCC_PLLP_DIV4;
      RCC_OscInitStruct.PLL.PLLQ = 2;
      RCC_OscInitStruct.PLL.PLLR = 2;
      if (HAL_RCC_OscConfig(&RCC_OscInitStruct) != HAL_OK)
      {
        Error_Handler();
      }
    
      /** Initializes the CPU, AHB and APB buses clocks
      */
      RCC_ClkInitStruct.ClockType = RCC_CLOCKTYPE_HCLK|RCC_CLOCKTYPE_SYSCLK
                                  |RCC_CLOCKTYPE_PCLK1|RCC_CLOCKTYPE_PCLK2;
      RCC_ClkInitStruct.SYSCLKSource = RCC_SYSCLKSOURCE_PLLCLK;
      RCC_ClkInitStruct.AHBCLKDivider = RCC_SYSCLK_DIV1;
      RCC_ClkInitStruct.APB1CLKDivider = RCC_HCLK_DIV2;
      RCC_ClkInitStruct.APB2CLKDivider = RCC_HCLK_DIV1;
    
      if (HAL_RCC_ClockConfig(&RCC_ClkInitStruct, FLASH_LATENCY_2) != HAL_OK)
      {
        Error_Handler();
      }
    }
    
    /**
      * @brief TIM1 Initialization Function
      * @param None
      * @retval None
      */
    static void MX_TIM1_Init(void)
    {
    
      /* USER CODE BEGIN TIM1_Init 0 */
    
      /* USER CODE END TIM1_Init 0 */
    
      TIM_MasterConfigTypeDef sMasterConfig = {0};
      TIM_OC_InitTypeDef sConfigOC = {0};
      TIM_BreakDeadTimeConfigTypeDef sBreakDeadTimeConfig = {0};
    
      /* USER CODE BEGIN TIM1_Init 1 */
    
      /* USER CODE END TIM1_Init 1 */
      htim1.Instance = TIM1;
      htim1.Init.Prescaler = 0;
      htim1.Init.CounterMode = TIM_COUNTERMODE_UP;
      htim1.Init.Period = 65535;
      htim1.Init.ClockDivision = TIM_CLOCKDIVISION_DIV1;
      htim1.Init.RepetitionCounter = 0;
      htim1.Init.AutoReloadPreload = TIM_AUTORELOAD_PRELOAD_DISABLE;
      if (HAL_TIM_PWM_Init(&htim1) != HAL_OK)
      {
        Error_Handler();
      }
      sMasterConfig.MasterOutputTrigger = TIM_TRGO_RESET;
      sMasterConfig.MasterSlaveMode = TIM_MASTERSLAVEMODE_DISABLE;
      if (HAL_TIMEx_MasterConfigSynchronization(&htim1, &sMasterConfig) != HAL_OK)
      {
        Error_Handler();
      }
      sConfigOC.OCMode = TIM_OCMODE_PWM1;
      sConfigOC.Pulse = 0;
      sConfigOC.OCPolarity = TIM_OCPOLARITY_HIGH;
      sConfigOC.OCNPolarity = TIM_OCNPOLARITY_HIGH;
      sConfigOC.OCFastMode = TIM_OCFAST_DISABLE;
      sConfigOC.OCIdleState = TIM_OCIDLESTATE_RESET;
      sConfigOC.OCNIdleState = TIM_OCNIDLESTATE_RESET;
      if (HAL_TIM_PWM_ConfigChannel(&htim1, &sConfigOC, TIM_CHANNEL_1) != HAL_OK)
      {
        Error_Handler();
      }
      if (HAL_TIM_PWM_ConfigChannel(&htim1, &sConfigOC, TIM_CHANNEL_4) != HAL_OK)
      {
        Error_Handler();
      }
      sBreakDeadTimeConfig.OffStateRunMode = TIM_OSSR_DISABLE;
      sBreakDeadTimeConfig.OffStateIDLEMode = TIM_OSSI_DISABLE;
      sBreakDeadTimeConfig.LockLevel = TIM_LOCKLEVEL_OFF;
      sBreakDeadTimeConfig.DeadTime = 0;
      sBreakDeadTimeConfig.BreakState = TIM_BREAK_DISABLE;
      sBreakDeadTimeConfig.BreakPolarity = TIM_BREAKPOLARITY_HIGH;
      sBreakDeadTimeConfig.AutomaticOutput = TIM_AUTOMATICOUTPUT_DISABLE;
      if (HAL_TIMEx_ConfigBreakDeadTime(&htim1, &sBreakDeadTimeConfig) != HAL_OK)
      {
        Error_Handler();
      }
      /* USER CODE BEGIN TIM1_Init 2 */
    
      /* USER CODE END TIM1_Init 2 */
      HAL_TIM_MspPostInit(&htim1);
    
    }
    
    /**
      * @brief TIM4 Initialization Function
      * @param None
      * @retval None
      */
    static void MX_TIM4_Init(void)
    {
    
      /* USER CODE BEGIN TIM4_Init 0 */
    
      /* USER CODE END TIM4_Init 0 */
    
      TIM_Encoder_InitTypeDef sConfig = {0};
      TIM_MasterConfigTypeDef sMasterConfig = {0};
    
      /* USER CODE BEGIN TIM4_Init 1 */
    
      /* USER CODE END TIM4_Init 1 */
      htim4.Instance = TIM4;
      htim4.Init.Prescaler = 0;
      htim4.Init.CounterMode = TIM_COUNTERMODE_UP;
      htim4.Init.Period = 65535;
      htim4.Init.ClockDivision = TIM_CLOCKDIVISION_DIV1;
      htim4.Init.AutoReloadPreload = TIM_AUTORELOAD_PRELOAD_DISABLE;
      sConfig.EncoderMode = TIM_ENCODERMODE_TI1;
      sConfig.IC1Polarity = TIM_ICPOLARITY_RISING;
      sConfig.IC1Selection = TIM_ICSELECTION_DIRECTTI;
      sConfig.IC1Prescaler = TIM_ICPSC_DIV1;
      sConfig.IC1Filter = 0;
      sConfig.IC2Polarity = TIM_ICPOLARITY_RISING;
      sConfig.IC2Selection = TIM_ICSELECTION_DIRECTTI;
      sConfig.IC2Prescaler = TIM_ICPSC_DIV1;
      sConfig.IC2Filter = 0;
      if (HAL_TIM_Encoder_Init(&htim4, &sConfig) != HAL_OK)
      {
        Error_Handler();
      }
      sMasterConfig.MasterOutputTrigger = TIM_TRGO_RESET;
      sMasterConfig.MasterSlaveMode = TIM_MASTERSLAVEMODE_DISABLE;
      if (HAL_TIMEx_MasterConfigSynchronization(&htim4, &sMasterConfig) != HAL_OK)
      {
        Error_Handler();
      }
      /* USER CODE BEGIN TIM4_Init 2 */
    
      /* USER CODE END TIM4_Init 2 */
    
    }
    
    /**
      * @brief TIM8 Initialization Function
      * @param None
      * @retval None
      */
    static void MX_TIM8_Init(void)
    {
    
      /* USER CODE BEGIN TIM8_Init 0 */
    
      /* USER CODE END TIM8_Init 0 */
    
      TIM_Encoder_InitTypeDef sConfig = {0};
      TIM_MasterConfigTypeDef sMasterConfig = {0};
    
      /* USER CODE BEGIN TIM8_Init 1 */
    
      /* USER CODE END TIM8_Init 1 */
      htim8.Instance = TIM8;
      htim8.Init.Prescaler = 0;
      htim8.Init.CounterMode = TIM_COUNTERMODE_UP;
      htim8.Init.Period = 65535;
      htim8.Init.ClockDivision = TIM_CLOCKDIVISION_DIV1;
      htim8.Init.RepetitionCounter = 0;
      htim8.Init.AutoReloadPreload = TIM_AUTORELOAD_PRELOAD_DISABLE;
      sConfig.EncoderMode = TIM_ENCODERMODE_TI1;
      sConfig.IC1Polarity = TIM_ICPOLARITY_RISING;
      sConfig.IC1Selection = TIM_ICSELECTION_DIRECTTI;
      sConfig.IC1Prescaler = TIM_ICPSC_DIV1;
      sConfig.IC1Filter = 0;
      sConfig.IC2Polarity = TIM_ICPOLARITY_RISING;
      sConfig.IC2Selection = TIM_ICSELECTION_DIRECTTI;
      sConfig.IC2Prescaler = TIM_ICPSC_DIV1;
      sConfig.IC2Filter = 0;
      if (HAL_TIM_Encoder_Init(&htim8, &sConfig) != HAL_OK)
      {
        Error_Handler();
      }
      sMasterConfig.MasterOutputTrigger = TIM_TRGO_RESET;
      sMasterConfig.MasterSlaveMode = TIM_MASTERSLAVEMODE_DISABLE;
      if (HAL_TIMEx_MasterConfigSynchronization(&htim8, &sMasterConfig) != HAL_OK)
      {
        Error_Handler();
      }
      /* USER CODE BEGIN TIM8_Init 2 */
    
      /* USER CODE END TIM8_Init 2 */
    
    }
    
    /**
      * @brief TIM12 Initialization Function
      * @param None
      * @retval None
      */
    static void MX_TIM12_Init(void)
    {
    
      /* USER CODE BEGIN TIM12_Init 0 */
    
      /* USER CODE END TIM12_Init 0 */
    
      TIM_OC_InitTypeDef sConfigOC = {0};
    
      /* USER CODE BEGIN TIM12_Init 1 */
    
      /* USER CODE END TIM12_Init 1 */
      htim12.Instance = TIM12;
      htim12.Init.Prescaler = 0;
      htim12.Init.CounterMode = TIM_COUNTERMODE_UP;
      htim12.Init.Period = 65535;
      htim12.Init.ClockDivision = TIM_CLOCKDIVISION_DIV1;
      htim12.Init.AutoReloadPreload = TIM_AUTORELOAD_PRELOAD_DISABLE;
      if (HAL_TIM_PWM_Init(&htim12) != HAL_OK)
      {
        Error_Handler();
      }
      sConfigOC.OCMode = TIM_OCMODE_PWM1;
      sConfigOC.Pulse = 0;
      sConfigOC.OCPolarity = TIM_OCPOLARITY_HIGH;
      sConfigOC.OCFastMode = TIM_OCFAST_DISABLE;
      if (HAL_TIM_PWM_ConfigChannel(&htim12, &sConfigOC, TIM_CHANNEL_1) != HAL_OK)
      {
        Error_Handler();
      }
      if (HAL_TIM_PWM_ConfigChannel(&htim12, &sConfigOC, TIM_CHANNEL_2) != HAL_OK)
      {
        Error_Handler();
      }
      /* USER CODE BEGIN TIM12_Init 2 */
    
      /* USER CODE END TIM12_Init 2 */
      HAL_TIM_MspPostInit(&htim12);
    
    }
    
    /**
      * @brief USART2 Initialization Function
      * @param None
      * @retval None
      */
    static void MX_USART2_UART_Init(void)
    {
    
      /* USER CODE BEGIN USART2_Init 0 */
    
      /* USER CODE END USART2_Init 0 */
    
      /* USER CODE BEGIN USART2_Init 1 */
    
      /* USER CODE END USART2_Init 1 */
      huart2.Instance = USART2;
      huart2.Init.BaudRate = 115200;
      huart2.Init.WordLength = UART_WORDLENGTH_8B;
      huart2.Init.StopBits = UART_STOPBITS_1;
      huart2.Init.Parity = UART_PARITY_NONE;
      huart2.Init.Mode = UART_MODE_TX_RX;
      huart2.Init.HwFlowCtl = UART_HWCONTROL_NONE;
      huart2.Init.OverSampling = UART_OVERSAMPLING_16;
      if (HAL_UART_Init(&huart2) != HAL_OK)
      {
        Error_Handler();
      }
      /* USER CODE BEGIN USART2_Init 2 */
    
      /* USER CODE END USART2_Init 2 */
    
    }
    
    /**
      * @brief GPIO Initialization Function
      * @param None
      * @retval None
      */
    static void MX_GPIO_Init(void)
    {
      GPIO_InitTypeDef GPIO_InitStruct = {0};
    /* USER CODE BEGIN MX_GPIO_Init_1 */
    /* USER CODE END MX_GPIO_Init_1 */
    
      /* GPIO Ports Clock Enable */
      __HAL_RCC_GPIOC_CLK_ENABLE();
      __HAL_RCC_GPIOH_CLK_ENABLE();
      __HAL_RCC_GPIOA_CLK_ENABLE();
      __HAL_RCC_GPIOB_CLK_ENABLE();
    
      /*Configure GPIO pin Output Level */
      HAL_GPIO_WritePin(LED_GPIO_Port, LED_Pin, GPIO_PIN_RESET);
    
      /*Configure GPIO pin : B1_Pin */
      GPIO_InitStruct.Pin = B1_Pin;
      GPIO_InitStruct.Mode = GPIO_MODE_IT_FALLING;
      GPIO_InitStruct.Pull = GPIO_NOPULL;
      HAL_GPIO_Init(B1_GPIO_Port, &GPIO_InitStruct);
    
      /*Configure GPIO pin : LED_Pin */
      GPIO_InitStruct.Pin = LED_Pin;
      GPIO_InitStruct.Mode = GPIO_MODE_OUTPUT_PP;
      GPIO_InitStruct.Pull = GPIO_NOPULL;
      GPIO_InitStruct.Speed = GPIO_SPEED_FREQ_LOW;
      HAL_GPIO_Init(LED_GPIO_Port, &GPIO_InitStruct);
    
    /* USER CODE BEGIN MX_GPIO_Init_2 */
    /* USER CODE END MX_GPIO_Init_2 */
    }
    
    /* USER CODE BEGIN 4 */
    
    /* USER CODE END 4 */
    
    /**
      * @brief  This function is executed in case of error occurrence.
      * @retval None
      */
    void Error_Handler(void)
    {
      /* USER CODE BEGIN Error_Handler_Debug */
      /* User can add his own implementation to report the HAL error return state */
      __disable_irq();
      while (1)
      {
      }
      /* USER CODE END Error_Handler_Debug */
    }
    
    #ifdef  USE_FULL_ASSERT
    /**
      * @brief  Reports the name of the source file and the source line number
      *         where the assert_param error has occurred.
      * @param  file: pointer to the source file name
      * @param  line: assert_param error line source number
      * @retval None
      */
    void assert_failed(uint8_t *file, uint32_t line)
    {
      /* USER CODE BEGIN 6 */
      /* User can add his own implementation to report the file name and line number,
         ex: printf("Wrong parameters value: file %s on line %d\r\n", file, line) */
      /* USER CODE END 6 */
    }
    #endif /* USE_FULL_ASSERT */

\end{lstlisting}

\subsubsection{エンコーダーライブラリ}

\begin{lstlisting}[language=C, caption=エンコーダーライブラリ(encoder.h)]
    // kinematics.h

    #ifndef KINEMATICS_H
    #define KINEMATICS_H
    
    typedef enum {
        OMNI_3,
        OMNI_4,
        MEKANUM
    } WheelMode;
    
    typedef struct {
        float robot_diameter;   // ロボットの直径
        float wheel_radius;     // ホイールの半径
        WheelMode mode;         // 動作モード
    } Kinematics;
    
    // 初期化関数
    void Kinematics_Init(Kinematics *kinematics, float robot_diameter, float wheel_radius, WheelMode mode);
    
    // モーターの目標速度を計算する関数
    void Kinematics_GetTargetSpeeds(Kinematics *kinematics, float lx, float ly, float rx, float *speedFR, float *speedFL, float *speedBR, float *speedBL);
    
    #endif // KINEMATICS_H
\end{lstlisting}

\begin{lstlisting}[language=C, caption=エンコーダーライブラリ(encoder.c)]
    #include "encoder.h"

    #define TIMER_MAX_COUNT 65535  // タイマーの最大値( 16 ビットタイマーの場合)
    
    void Encoder_Init(Encoder* encoder, TIM_HandleTypeDef* htim, double diameter, int ppr, int period)
    {
        encoder->htim = htim;
        encoder->ppr = ppr;
        encoder->diameter = diameter;
        encoder->period = period;
        encoder->limit = 0;
        encoder->before_rot = 0.0;
    
        encoder->htim->Init.Prescaler = 0;
        encoder->htim->Init.CounterMode = TIM_COUNTERMODE_UP;
        encoder->htim->Init.Period = TIMER_MAX_COUNT;
        encoder->htim->Init.ClockDivision = TIM_CLOCKDIVISION_DIV1;
    
        TIM_Encoder_InitTypeDef encoder_init;
        encoder_init.EncoderMode = TIM_ENCODERMODE_TI12;
        encoder_init.IC1Polarity = TIM_ICPOLARITY_RISING;
        encoder_init.IC1Selection = TIM_ICSELECTION_DIRECTTI;
        encoder_init.IC1Prescaler = TIM_ICPSC_DIV1;
        encoder_init.IC1Filter = 0;
        encoder_init.IC2Polarity = TIM_ICPOLARITY_RISING;
        encoder_init.IC2Selection = TIM_ICSELECTION_DIRECTTI;
        encoder_init.IC2Prescaler = TIM_ICPSC_DIV1;
        encoder_init.IC2Filter = 0;
    
        HAL_TIM_Encoder_Init(htim, &encoder_init);
        HAL_TIM_Encoder_Start(htim, TIM_CHANNEL_ALL);
        __HAL_TIM_SET_COUNTER(htim, TIMER_MAX_COUNT / 2);  // カウンタを中央に設定
    }
    
    int Encoder_Read(Encoder* encoder)
    {
        int16_t count = (int16_t)(__HAL_TIM_GET_COUNTER(encoder->htim) - TIMER_MAX_COUNT / 2);
        return count;
    }
    
    void Encoder_Interrupt(Encoder* encoder, EncoderData* encoder_data)
    {
        int count = Encoder_Read(encoder);
    
        encoder_data->count = count;
        encoder_data->rot = count / (double)encoder->ppr;
        encoder_data->deg = encoder_data->rot * 360.0;
        encoder_data->distance = encoder_data->rot * (PI * encoder->diameter);
    
        encoder_data->rps = (encoder_data->rot - encoder->before_rot) / (encoder->period * 0.001);
        encoder_data->velocity = encoder_data->rps * PI * encoder->diameter;
    
        encoder->before_rot = encoder_data->rot;
    }
    
    void Encoder_Reset(Encoder* encoder)
    {
        __HAL_TIM_SET_COUNTER(encoder->htim, TIMER_MAX_COUNT / 2);  // カウンタを中央にリセット
    }
\end{lstlisting}

\subsubsection{モータドライバーライブラリ}

\begin{lstlisting}[language=C, caption=モータドライバーライブラリ(motor\_driver.h)]
    #ifndef MOTOR_DRIVER_H
    #define MOTOR_DRIVER_H
    
    #include "stm32f4xx_hal.h"
    
    typedef struct {
        TIM_HandleTypeDef* htimA;  // タイマーA
        uint32_t channelA;         // タイマーチャンネルA
        TIM_HandleTypeDef* htimB;  // タイマーB
        uint32_t channelB;         // タイマーチャンネルB
    } MotorDriver;
    
    // モータードライバを初期化する関数
    void MotorDriver_Init(MotorDriver* motor, TIM_HandleTypeDef* htimA, uint32_t channelA,
                          TIM_HandleTypeDef* htimB, uint32_t channelB);
    
    // モーターの速度を設定する関数
    void MotorDriver_setSpeed(MotorDriver* motor, int speed);
    
    #endif /* MOTOR_DRIVER_H */
\end{lstlisting}

\begin{lstlisting}[language=C, caption=モータドライバーライブラリ(motor\_driver.c)]
    #include "motor_driver.h"

    // 初期化関数
    void MotorDriver_Init(MotorDriver* motor, TIM_HandleTypeDef* htimA, uint32_t channelA,
                          TIM_HandleTypeDef* htimB, uint32_t channelB) {
        motor->htimA = htimA;
        motor->channelA = channelA;
        motor->htimB = htimB;
        motor->channelB = channelB;
    
        // PWM 開始
        HAL_TIM_PWM_Start(htimA, channelA);
        HAL_TIM_PWM_Start(htimB, channelB);
    }
    
    // 速度設定関数
    void MotorDriver_setSpeed(MotorDriver *motor, int speed) {
        int pwm_value;
        if (speed > 100) speed = 99;
        if (speed < -100) speed = -99;
    
        if (speed > 0) {
            pwm_value = (speed * __HAL_TIM_GET_AUTORELOAD(motor->htimA)) / 100;
            __HAL_TIM_SET_COMPARE(motor->htimA, motor->channelA, pwm_value);
            __HAL_TIM_SET_COMPARE(motor->htimB, motor->channelB, 0);
        } else {
            pwm_value = (-speed * __HAL_TIM_GET_AUTORELOAD(motor->htimA)) / 100;
            __HAL_TIM_SET_COMPARE(motor->htimA, motor->channelA, 0);
            __HAL_TIM_SET_COMPARE(motor->htimB, motor->channelB, pwm_value);
        }
    }
\end{lstlisting}

\subsubsection{シリアル通信ライブラリ}

\begin{lstlisting}[language=C, caption=シリアル通信ライブラリ(serial\_lib.h)]
    #ifndef SERIAL_LIB_H
    #define SERIAL_LIB_H
    
    #include "main.h"
    
    // シリアルヘッダーの定義
    #define SERIAL_HEADER1 0xA5
    #define SERIAL_HEADER2 0xA5
    
    // 関数プロトタイプ
    void Serial_Init(UART_HandleTypeDef *huart);
    void Serial_SendData(UART_HandleTypeDef *huart, int16_t *data, uint8_t data_count);
    uint8_t Serial_ReceiveData(UART_HandleTypeDef *huart, int16_t *data, uint8_t data_count);
    
    #endif // SERIAL_LIB_H
\end{lstlisting}

\begin{lstlisting}[language=C, caption=シリアル通信ライブラリ(serial\_lib.c)]
    #include "serial_lib.h"
    #include <stdlib.h>
    
    // シリアル通信の初期化
    void Serial_Init(UART_HandleTypeDef *huart) {
        HAL_UART_Init(huart);
    }
    
    // 可変長データの送信関数
    void Serial_SendData(UART_HandleTypeDef *huart, int16_t *data, uint8_t data_count) {
        uint8_t buffer_size = 2 + data_count * 2;
        uint8_t *buffer = (uint8_t *)malloc(buffer_size);
    
        buffer[0] = SERIAL_HEADER1;
        buffer[1] = SERIAL_HEADER2;
    
        for (uint8_t i = 0; i < data_count; i++) {
            buffer[2 + i * 2] = (data[i] >> 8) & 0xFF;
            buffer[3 + i * 2] = data[i] & 0xFF;
        }
    
        HAL_UART_Transmit(huart, buffer, buffer_size, HAL_MAX_DELAY);
        free(buffer);
    }
    
    // 可変長データの受信関数
    uint8_t Serial_ReceiveData(UART_HandleTypeDef *huart, int16_t *data, uint8_t data_count) {
        uint8_t buffer_size = 2 + data_count * 2;
        uint8_t *buffer = (uint8_t *)malloc(buffer_size);
    
        if (HAL_UART_Receive(huart, buffer, buffer_size, HAL_MAX_DELAY) == HAL_OK) {
            if (buffer[0] == SERIAL_HEADER1 && buffer[1] == SERIAL_HEADER2) {
                for (uint8_t i = 0; i < data_count; i++) {
                    data[i] = (buffer[2 + i * 2] << 8) | buffer[3 + i * 2];
                }
                free(buffer);
                return 1; // 正常受信
            }
        }
        free(buffer);
        return 0; // エラー
    }
\end{lstlisting}

%%%%%%%%%%%%%%%%%%%%%%%%%%%%%%%%%%%%%%%%%%%%%%%%%%%%%%
\subsection{上位レイヤープログラム}
%%%%%%%%%%%%%%%%%%%%%%%%%%%%%%%%%%%%%%%%%%%%%%%%%%%%%%

作成したプログラムを以下に示す.
また,\url{https://github.com/Altairu/Person_Tracking_Roboware}に掲載している.

\begin{lstlisting}[language=Python, caption=PID\_node.py]
    import rclpy
    from rclpy.node import Node
    from std_msgs.msg import Float32MultiArray
    
    class PIDController:
        def __init__(self, kp, ki, kd):
            self.kp = kp
            self.ki = ki
            self.kd = kd
            self.prev_error = 0.0
            self.integral = 0.0
    
        def compute(self, target, current, dt):
            if target == 0.0 and current == 0.0:
                self.reset()
                return 0.0  # 特別条件: 目標値と現在値が0の場合は出力も0
    
            error = target - current
            self.integral += error * dt
            derivative = (error - self.prev_error) / dt
            self.prev_error = error
    
            return self.kp * error + self.ki * self.integral + self.kd * derivative
    
        def reset(self):
            self.prev_error = 0.0
            self.integral = 0.0
    
    class PIDNode(Node):
        def __init__(self):
            super().__init__('PID_node')
    
            self.create_subscription(Float32MultiArray, 'wheel_targets', self.target_callback, 10)
            self.create_subscription(Float32MultiArray, 'wheel_feedback', self.feedback_callback, 10)
            self.pub = self.create_publisher(Float32MultiArray, 'wheel_controls', 10)
    
            self.pid_right = PIDController(kp=0.003, ki=0.00005, kd=0.0)
            self.pid_left = PIDController(kp=0.003, ki=0.00005, kd=0.0)
    
            self.target_right = 0.0
            self.target_left = 0.0
            self.current_right = 0.0
            self.current_left = 0.0
    
            self.create_timer(0.1, self.control_loop)
            self.last_time = self.get_clock().now()
    
        def target_callback(self, msg):
            if len(msg.data) == 2:
                self.target_right = msg.data[0]
                self.target_left = msg.data[1]
            else:
                self.get_logger().error(f"Invalid data received in 'wheel_targets'. Expected 2 floats, got {len(msg.data)}")
    
        def feedback_callback(self, msg):
            if len(msg.data) == 2:
                self.current_right = msg.data[0]
                self.current_left = msg.data[1]
            else:
                self.get_logger().error(f"Invalid data received in 'wheel_feedback'. Expected 2 floats, got {len(msg.data)}")
    
        def control_loop(self):
            current_time = self.get_clock().now()
            dt = (current_time - self.last_time).nanoseconds / 1e9
            self.last_time = current_time
    
            control_signal_right = self.pid_right.compute(self.target_right, self.current_right, dt)
            control_signal_left = self.pid_left.compute(self.target_left, self.current_left, dt)
    
            # 制御信号が有効な範囲内か検証
            if not (self.is_valid_float(control_signal_right) and self.is_valid_float(control_signal_left)):
                self.get_logger().error("Control signals contain invalid values. Skipping this loop.")
                return
    
            # 制御信号をパブリッシュ
            control_msg = Float32MultiArray()
            control_msg.data = [float(control_signal_right), float(control_signal_left)]
            self.pub.publish(control_msg)
    
            # デバッグ情報を出力
            self.get_logger().info(
                f"Target: Right={self.target_right}, Left={self.target_left} | "
                f"Current: Right={self.current_right}, Left={self.current_left} | "
                f"Control: Right={control_signal_right}, Left={control_signal_left}"
            )
    
        def is_valid_float(self, value):
            try:
                float_value = float(value)
                return float('-inf') < float_value < float('inf')  # 有限数の検証
            except ValueError:
                return False
    
    def main(args=None):
        rclpy.init(args=args)
        node = PIDNode()
        rclpy.spin(node)
        rclpy.shutdown()
    
    if __name__ == '__main__':
        main()
\end{lstlisting}

\begin{lstlisting}[language=Python, caption=RealSense\_node.py]
    import rclpy
    from rclpy.node import Node
    from std_msgs.msg import Float32MultiArray
    import pyrealsense2 as rs
    import numpy as np
    import torch
    import cv2
    
    class RealSenseNode(Node):
        def __init__(self):
            super().__init__('realsense_node')
            self.publisher = self.create_publisher(Float32MultiArray, 'camera_data', 10)
    
            # RealSense設定
            self.pipeline = rs.pipeline()
            config = rs.config()
            config.enable_stream(rs.stream.depth, 640, 480, rs.format.z16, 30)
            config.enable_stream(rs.stream.color, 640, 480, rs.format.bgr8, 30)
            self.pipeline.start(config)
    
            # YOLOモデルロード
            self.model = torch.hub.load('/home/altair/Roboware/ultralytics/yolov5', 'custom', 
                                        path='/home/altair/Roboware/ultralytics/yolov5/yolov5s.pt', 
                                        source='local')
            self.create_timer(0.1, self.process_frames)
    
            # フィルタ用変数
            self.previous_distance = 0.0  # 前回の距離値
            self.distance_history = []  # 距離履歴
            self.history_size = 5  # 移動平均履歴の最大サイズ
    
        def process_frames(self):
            try:
                frames = self.pipeline.wait_for_frames()
                depth_frame = frames.get_depth_frame()
                color_frame = frames.get_color_frame()
                if not depth_frame or not color_frame:
                    return
    
                depth_image = np.asanyarray(depth_frame.get_data())
                color_image = np.asanyarray(color_frame.get_data())
                results = self.model(color_image)
    
                for result in results.xyxy[0]:  # 検出結果をループ
                    box, conf, cls = result[:4], result[4], int(result[5])
                    if cls == 0:  # クラス0(人物)のみ処理
                        x1, y1, x2, y2 = map(int, box)
                        center_x, center_y = (x1 + x2) // 2, (y1 + y2) // 2
    
                        raw_distance = depth_frame.get_distance(center_x, center_y)
                        offset_x = center_x - (color_image.shape[1] // 2)
    
                        # 距離の処理
                        filtered_distance = self.filter_distance(raw_distance)
    
                        # Publish camera data
                        msg = Float32MultiArray()
                        msg.data = [filtered_distance, float(offset_x)]
                        self.publisher.publish(msg)
    
                        self.get_logger().info(f"Published camera data: Distance={filtered_distance:.2f}, Offset={offset_x}")
    
                        # 画面に検出結果を描画
                        cv2.rectangle(color_image, (x1, y1), (x2, y2), (0, 255, 0), 2)
                        cv2.putText(color_image, f"Distance: {filtered_distance:.2f}m", (x1, y1 - 10),
                                    cv2.FONT_HERSHEY_SIMPLEX, 0.5, (0, 255, 0), 2)
                        cv2.putText(color_image, f"Offset: {offset_x}", (x1, y1 - 30),
                                    cv2.FONT_HERSHEY_SIMPLEX, 0.5, (0, 255, 0), 2)
    
                        self.previous_distance = filtered_distance  # 前回の距離を更新
                        break
    
                # カメラ映像を表示
                cv2.imshow("RealSense Detection", color_image)
                cv2.waitKey(1)
    
            except Exception as e:
                self.get_logger().error(f"Error processing frames: {str(e)}")
    
        def filter_distance(self, current_distance):
            """
            距離が近づく場合はそのまま、遠ざかる場合はフィルタリングする。
            """
            if current_distance == 0.0:  # 無効値は無視
                return self.previous_distance
    
            if current_distance < self.previous_distance:
                # 近づいている場合: そのままの値を使用
                self.distance_history = [current_distance]  # 履歴をリセット
                return current_distance
            else:
                # 遠ざかる場合: 移動平均フィルタを適用
                self.distance_history.append(current_distance)
                if len(self.distance_history) > self.history_size:
                    self.distance_history.pop(0)  # 古い値を削除
                return sum(self.distance_history) / len(self.distance_history)
    
        def destroy_node(self):
            self.pipeline.stop()
            cv2.destroyAllWindows()
            super().destroy_node()
    
    def main(args=None):
        rclpy.init(args=args)
        node = RealSenseNode()
        rclpy.spin(node)
        node.destroy_node()
        rclpy.shutdown()
    
    if __name__ == '__main__':
        main()
\end{lstlisting}

\begin{lstlisting}[language=Python, caption=Roboware\_node.py]
    import rclpy
    from rclpy.node import Node
    from std_msgs.msg import String, Float32MultiArray
    import csv
    import os
    import math
    
    class RobowareNode(Node):
        def __init__(self):
            super().__init__('Roboware_node')
            self.subscription = self.create_subscription(
                String,
                'web_socket_pub',
                self.listener_callback,
                10
            )
            self.position_subscription = self.create_subscription(
                Float32MultiArray,
                'camera_data',
                self.position_callback,
                10
            )
            self.publisher = self.create_publisher(Float32MultiArray, 'wheel_targets', 10)
    
            self.mode = 0  # Initial mode (Control: 0, Follow: 1)
            self.target_right = 0.0
            self.target_left = 0.0
            self.person_distance = 0.0
            self.person_offset = 0.0
            self.previous_offset = 0.0  # 前回のオフセット
            self.kp_v = 5000.0  # Proportional gain for velocity
            self.kp_omega = 50.0  # Proportional gain for angular velocity
            self.kd_lambda = 0.1  # 最大微分ゲイン
            self.a = 0.1  # 動的微分ゲイン調整パラメータ
            self.navigation_constant = 2.0  # Proportional navigation constant (N)
            self.dt = 0.1  # サンプリング間隔(秒)
    
            # CSVファイルの設定
            self.csv_file = "new_MPN_data.csv"
            self.initialize_csv()
    
            # タイマー設定
            self.timer = self.create_timer(0.1, self.record_data)  # 0.1 秒ごとに実行
            self.recording = False  # 記録中フラグ
    
        def initialize_csv(self):
            """CSV ファイルを初期化し、ヘッダーを記録"""
            if not os.path.exists(self.csv_file):
                with open(self.csv_file, mode='w', newline='') as file:
                    writer = csv.writer(file)
                    writer.writerow([
                        'Time', 'PersonDistance', 'PersonOffset', 'V', 'Omega', 
                        'TargetRight', 'TargetLeft'
                    ])
    
        def save_to_csv(self, time, distance, offset, v, omega, right_speed, left_speed):
            """データを CSV ファイルに追記"""
            with open(self.csv_file, mode='a', newline='') as file:
                writer = csv.writer(file)
                writer.writerow([time, distance, offset, v, omega, right_speed, left_speed])
    
        def listener_callback(self, msg):
            try:
                data = list(map(float, msg.data.split(',')))
                if len(data) != 6:
                    self.get_logger().warn("Received invalid data length.")
                    return
    
                mode, rx, ry, lx, ly, stop = data
                self.mode = int(mode)
    
                if stop == 1:
                    self.target_right = 0.0
                    self.target_left = 0.0
                    self.recording = False  # 記録停止
                    self.get_logger().info("Emergency stop activated.")
                elif self.mode == 0:  # Control Mode (手動モード)
                    V = float((int(ry) - 106) * 200)
                    omega = float(-1 * (int(lx) - 102) * 100)
                    self.target_right = V + omega
                    self.target_left = V - omega
                    self.recording = False  # 記録停止
                elif self.mode == 1:  # Follow Mode (画像処理モード)
                    self.follow_person()
                    self.recording = True  # 記録開始
    
                # Publish the wheel targets
                self.publish_targets()
    
            except ValueError as e:
                self.get_logger().error(f"Error parsing message: {e}")
    
        def position_callback(self, msg):
            if len(msg.data) == 2:  # Assuming message contains [distance, offset]
                self.person_distance = msg.data[0]
                self.person_offset = msg.data[1]
                self.get_logger().info(f"Updated position: Distance={self.person_distance}, Offset={self.person_offset}")
            else:
                self.get_logger().warn("Invalid position data received.")
    
        def follow_person(self):
            # 修正比例航法 ( MPN )
            distance = max(self.person_distance, 1.0)
            offset_rate = (self.person_offset - self.previous_offset) / self.dt  # 偏差角速度
    
            # 動的微分ゲイン
            dynamic_kd = self.kd_lambda * (1 - math.exp(-self.a * abs(offset_rate))) / (1 + math.exp(-self.a * abs(offset_rate)))
    
            # 直進速度
            V = self.kp_v * (distance - 1.0)
    
            # 修正された角速度
            omega = (
                -1 * self.navigation_constant * self.kp_omega * self.person_offset / distance +
                dynamic_kd * offset_rate
            )
    
            V = max(min(V, 50000.0), -50000.0)  # Max forward/backward velocity
            omega = max(min(omega, 30000.0), -30000.0)  # Max rotational velocity
    
            self.current_v = V
            self.current_omega = omega
    
            # Calculate target velocities for left and right wheels
            self.target_right = V + omega
            self.target_left = V - omega
    
            # 前回の偏差を更新
            self.previous_offset = self.person_offset
    
            self.get_logger().info(f"Follow mode | Distance={self.person_distance}, Offset={self.person_offset}, OffsetRate={offset_rate} | Kd={dynamic_kd:.2f} | V={V}, Omega={omega} | Target Right={self.target_right}, Left={self.target_left}")
    
        def publish_targets(self):
            msg = Float32MultiArray()
            msg.data = [self.target_right, self.target_left]
            self.publisher.publish(msg)
    
        def record_data(self):
            """0.1 秒ごとにデータを CSV に記録"""
            if self.recording:  # 記録中のみデータを保存
                current_time = self.get_clock().now().to_msg().sec  # 現在時刻を秒単位で取得
                self.save_to_csv(
                    current_time, 
                    self.person_distance, 
                    self.person_offset, 
                    self.current_v, 
                    self.current_omega, 
                    self.target_right, 
                    self.target_left
                )
    
    def main(args=None):
        rclpy.init(args=args)
        node = RobowareNode()
        rclpy.spin(node)
        node.destroy_node()
        rclpy.shutdown()
    
    if __name__ == '__main__':
        main()
\end{lstlisting}

\begin{lstlisting}[language=Python, caption=serial\_read\_node.py]
    import rclpy
    from rclpy.node import Node
    from std_msgs.msg import Float32MultiArray
    import serial
    import struct
    
    class SerialReadNode(Node):
        def __init__(self):
            super().__init__('serial_read_node')
    
            # シリアルポート設定
            self.serial_port = "/dev/ttyACM0"
            self.baudrate = 115200
            self.timeout = 0.01
    
            try:
                self.ser = serial.Serial(self.serial_port, self.baudrate, timeout=self.timeout)
                self.get_logger().info(f"Connected to serial port: {self.serial_port}")
            except serial.SerialException as e:
                self.get_logger().error(f"Failed to connect to serial port: {e}")
                self.ser = None
    
            # パブリッシャ設定
            self.publisher = self.create_publisher(Float32MultiArray, 'wheel_feedback', 10)
    
            # タイマーで定期的にデータを読む
            self.timer = self.create_timer(0.01, self.read_from_serial)
    
        def read_from_serial(self):
            if self.ser is not None and self.ser.in_waiting >= 6:
                try:
                    received_data = self.ser.read(6)
                    if len(received_data) == 6 and received_data[0] == 0xA5 and received_data[1] == 0xA5:
                        _, _, right_speed, left_speed = struct.unpack('>BBhh', received_data)
                        msg = Float32MultiArray()
                        msg.data = [float(right_speed), float(left_speed)]
                        self.publisher.publish(msg)
    
                        # デバッグログ
                        self.get_logger().info(f"Received from serial: Right={right_speed}, Left={left_speed}")
                    else:
                        self.get_logger().error("Invalid data format received from serial.")
                except Exception as e:
                    self.get_logger().error(f"Error reading data from serial: {e}")
    
    def main(args=None):
        rclpy.init(args=args)
        node = SerialReadNode()
        rclpy.spin(node)
        rclpy.shutdown()
    
    if __name__ == '__main__':
        main()
\end{lstlisting}

\begin{lstlisting}[language=Python, caption=serial\_send\_node.py]
    import rclpy
    from rclpy.node import Node
    from std_msgs.msg import Float32MultiArray
    import serial
    import struct
    
    class SerialSendNode(Node):
        def __init__(self):
            super().__init__('serial_send_node')
    
            # シリアルポート設定
            self.serial_port = "/dev/ttyACM0"
            self.baudrate = 115200
            self.timeout = 0.01
    
            try:
                self.ser = serial.Serial(self.serial_port, self.baudrate, timeout=self.timeout)
                self.get_logger().info(f"Connected to serial port: {self.serial_port}")
            except serial.SerialException as e:
                self.get_logger().error(f"Failed to connect to serial port: {e}")
                self.ser = None
    
            # サブスクライバ設定
            self.subscription = self.create_subscription(
                Float32MultiArray,
                'wheel_controls',
                self.control_callback,
                10
            )
    
        def control_callback(self, msg):
            if self.ser is not None:
                try:
                    if len(msg.data) == 2:
                        right_control, left_control = msg.data
                        send_data = struct.pack('>BBhh', 0xA5, 0xA5, int(right_control), int(left_control))
                        self.ser.write(send_data)
    
                        # デバッグログ
                        self.get_logger().info(f"Sent to serial: Right={right_control}, Left={left_control}")
                    else:
                        self.get_logger().error("Invalid control data length. Expected 2 values.")
                except Exception as e:
                    self.get_logger().error(f"Error sending data to serial: {e}")
    
    def main(args=None):
        rclpy.init(args=args)
        node = SerialSendNode()
        rclpy.spin(node)
        rclpy.shutdown()
    
    if __name__ == '__main__':
        main()
\end{lstlisting}

\begin{lstlisting}[language=Python, caption=web\_socket\_node.py]
    import threading
    import rclpy
    from rclpy.node import Node
    from std_msgs.msg import String, Float32MultiArray
    
    from fastapi import FastAPI, WebSocket as FastAPIWebSocket
    from fastapi.responses import HTMLResponse
    import uvicorn
    import os
    from socket import SO_REUSEADDR, SOL_SOCKET, socket
    
    
    # IPアドレスとポートの指定
    ipadress_ = '192.168.11.47'
    port_ = 8080
    
    # HTMLファイルのパス
    path = '/home/altair/Roboware/UI.txt'
    
    # FastAPIのインスタンスを作成
    app = FastAPI()
    
    # HTMLファイルが存在するか確認し、読み込み
    if not os.path.exists(path):
        raise FileNotFoundError(f'File not found: {path}')
    
    with open(path, 'r') as f:
        html = f.read()
    
    # ROS 2 ノードの定義
    class WebSocketNode(Node):
        def __init__(self):
            msg = String()
            super().__init__('web_socket_node')
            self.send_data = ''
    
            # パブリッシャーを作成
            self.pub = self.create_publisher(String, 'web_socket_pub', 10)
            
            # サブスクリプションを作成し、コールバック関数を設定
            self.sub = self.create_subscription(Float32MultiArray, 'estimated_position', self.callback, 10)
    
            # FastAPIルートの定義
            @app.get("/")
            async def get():
                return HTMLResponse(html)
            
            # WebSocketエンドポイントの定義
            @app.websocket('/ws')
            async def websocket_endpoint(websocket: FastAPIWebSocket):
                await websocket.accept()
                try:
                    while True:
                        # クライアントからのデータを受信
                        receive_data = await websocket.receive_text()
                        
                        # 受信したデータをROSトピックにパブリッシュ
                        msg.data = receive_data
                        self.pub.publish(msg)
                        print(msg.data)
                        # サブスクライブしたデータをクライアントに送信
                        string_send_data = ",".join(map(str, self.send_data))
                        await websocket.send_text(string_send_data)
                except Exception as e:
                    print(f'WebSocket error: {str(e)}')
                finally:
                    print('WebSocket disconnected')
    
        # サブスクリプションのコールバック関数
        def callback(self, sub_msg):
            self.send_data = sub_msg.data
    
    # ROS 2ノードを実行する関数
    def run_ros2():
        rclpy.init()
        node = WebSocketNode()
        rclpy.spin(node)
        rclpy.shutdown()
    
    # FastAPIサーバーを実行する関数
    def run_fastapi():
        # Uvicornの設定を作成
        config = uvicorn.Config(app, host=ipadress_, port=port_, log_level="info")
    
        # Uvicornサーバーを作成
        server = uvicorn.Server(config)
    
        # FastAPIサーバーを起動
        server.run()
    
    # メイン関数
    def main():
        # ROS 2のspinをメインスレッドで実行
        ros2_thread = threading.Thread(target=run_ros2)
        ros2_thread.start()
    
        # FastAPIサーバーを別のスレッドで実行
        fastapi_thread = threading.Thread(target=run_fastapi)
        fastapi_thread.start()
    
        # 両方のスレッドが終了するのを待つ
        ros2_thread.join()
        fastapi_thread.join()
    
    if __name__ == '__main__':
        main()
\end{lstlisting}


\begin{lstlisting}[language=HTML, caption=UI.txt]
    <!DOCTYPE html>
    <html lang="en">
    <head>
        <meta charset="UTF-8">
        <meta name="viewport" content="width=device-width, initial-scale=1.0">
        <title>Robot Control</title>
        <style>
            body {
                font-family: Arial, sans-serif;
                text-align: center;
                margin: 0;
                padding: 0;
            }
            .container {
                padding: 20px;
            }
            .button {
                width: 200px;
                height: 50px;
                margin: 10px;
                font-size: 16px;
                font-weight: bold;
                background-color: #007bff;
                color: white;
                border: none;
                border-radius: 5px;
                cursor: pointer;
            }
            .button:active {
                background-color: #0056b3;
            }
            .toggle-button {
                width: 200px;
                height: 50px;
                font-size: 16px;
                font-weight: bold;
                margin: 20px auto;
                background-color: #ff5722;
                color: white;
                border: none;
                border-radius: 5px;
                cursor: pointer;
            }
            .toggle-button.active {
                background-color: #4caf50;
            }
            .status {
                margin-top: 20px;
                font-size: 18px;
            }
            .status.connected {
                color: green;
            }
            .status.disconnected {
                color: red;
            }
        </style>
    </head>
    <body>
        <div class="container">
            <h1>Robot Control Panel</h1>
            <button class="toggle-button" id="modeToggle" onclick="toggleMode()">Mode: Control</button>
            <div class="status" id="status">Status: Disconnected</div>
        </div>
    
        <script>
            let websocket;
            let currentMode = 0; // Control: 0, Follow: 1
            let stop = 0; // Emergency stop: 1 = Stop, 0 = Running
            let lx = 100, ly = 100, rx = 100, ry = 100; // Joystick default values
            const reconnectInterval = 5000;
    
            function connectWebSocket() {
                websocket = new WebSocket("ws://192.168.11.47:8080/ws");
    
                websocket.onopen = function () {
                    document.getElementById('status').innerText = "Status: Connected";
                    document.getElementById('status').className = "status connected";
                };
    
                websocket.onclose = function () {
                    document.getElementById('status').innerText = "Status: Disconnected";
                    document.getElementById('status').className = "status disconnected";
                    setTimeout(connectWebSocket, reconnectInterval);
                };
    
                websocket.onerror = function (error) {
                    console.error("WebSocket error:", error);
                };
            }
    
            function toggleMode() {
                const modeButton = document.getElementById("modeToggle");
                currentMode = currentMode === 0 ? 1 : 0;
    
                if (currentMode === 1) {
                    modeButton.innerText = "Mode: Follow";
                    modeButton.classList.add("active");
                } else {
                    modeButton.innerText = "Mode: Control";
                    modeButton.classList.remove("active");
                }
    
                sendCommand();
            }
    
            function updateGamepadValues() {
                const gamepads = navigator.getGamepads();
                const gamepad = gamepads[0];
    
                if (gamepad) {
                    lx = gamepad.axes[0] * 100 + 100;
                    ly = gamepad.axes[1] * -100 + 100;
                    rx = gamepad.axes[2] * 100 + 100;
                    ry = gamepad.axes[3] * -100 + 100;
    
                    // Send updated joystick values
                    sendCommand();
                }
            }
    
            function sendCommand() {
                if (websocket && websocket.readyState === WebSocket.OPEN) {
                    const command = [currentMode, rx, ry, lx, ly, stop];
                    websocket.send(command.join(","));
                }
            }
    
            function setupGamepad() {
                setInterval(updateGamepadValues, 100);
            }
    
            document.addEventListener('contextmenu', event => event.preventDefault());
            connectWebSocket();
            setupGamepad();
        </script>
    </body>
    </html>
\end{lstlisting}

\newpage
%%%%%%%%%%%%%%%%%%%%%%%%%%%%%%%%%%%%%%%%%%%%%%%%%%%%%%
\subsection{回路}
%%%%%%%%%%%%%%%%%%%%%%%%%%%%%%%%%%%%%%%%%%%%%%%%%%%%%%
作成した回路を以下に示す.
また,\url{https://github.com/Altairu/AltairMD_V7},
\url{https://github.com/Altairu/ALTAIR_MDD_V2}に掲載している.

\begin{figure}[h]
    \centering
    \includegraphics[width=0.67\textwidth]{figure/MDDv2.pdf}
    \caption{モータードライバードライバーV2 回路図}
\end{figure}

\begin{figure}[h]
    \centering
    \includegraphics[width=0.67\textwidth]{figure/MDV7.pdf}
    \caption{モータードライバーV7 回路図}
\end{figure}
% !TEX root = main.tex
\begin{center}
    \section*{論\,文\,要\,旨}                      %% ここに番号をつけない
\end{center}

近年,生産年齢人口の減少が社会的な課題となっており,
ロボット技術の導入による作業の効率化が期待されている.
例えば,草刈りロボットや運搬ロボットなど,人手不足を補うための自律ロボットが徐々に普及しつつある.
しかしながら,多くの自動搬送ロボットにはLiDARや3Dセンサーなどの高価なセンサーが搭載されているケースが多い.
LiDARは高精度な距離計測を可能にし,障害物回避や人追従において重要な役割を果たす一方で,その導入コストが非常に高く,
特に中小企業や農業現場では導入が難しい現状がある.
近年,カメラ映像を用いて物体位置を認識し,自律移動を行うシステムは物流や農業,介護など幅広い分野で需要が高まっている.

本研究では,カメラのみを用いた低コストな人追従搬送ロボットの開発を目指し,
Intel RealSense D435iを使用した画像処理技術と追従制御アルゴリズムの実装に取り組んだ.

具体的には,デプスカメラから取得した深度データを用いて,画像上の座標を実空間の物理座標に変換し,人の位置を推定する.
また,追従制御には,比例航法(PN),修正比例航法(MPN),およびゲインスケジューリング下修正比例航法(GS-MPN)の3つのアルゴリズムを実装し,
それぞれの性能を比較評価した.さらに,ROS2を用いたシステム全体の統合を行い,上位層と下位層を連携させたリアルタイム制御を実現した.

実験では,ロボットが直線および曲線の軌道を持つ目標物を追従する際の精度と滑らかさを評価した.
結果として,GS-MPNが最もバランスの取れた性能を示し,急加速を抑えつつ安定した追従を実現した.
これにより,従来手法で見られた不安定な動作や追従精度の低下が大幅に改善された.
一方で,遮蔽物や複数対象物が存在する場合の課題が残されており,これらの状況に対応するための識別技術や外乱耐性の強化が今後の課題である.

本研究の成果は,コスト効率に優れた人追従ロボットの実現に向けた重要なステップとなり,
物流や医療分野など多様な応用への可能性を示唆するものである.
今後は,環境適応性の向上や予測制御技術の導入を通じて,さらに高性能で信頼性の高いロボットシステムの開発を目指す.
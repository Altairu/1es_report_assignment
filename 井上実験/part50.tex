% !TEX root = main.tex
\section{まとめ}

本実験では,Web技術とデータベース技術を組み合わせたWebデータベースの構築を段階的に実施した.

実験Ⅰでは,XAMPPを用いてApache,MySQL(MariaDB),PHPなどの環境をUbuntu上に構築し,
Webサーバの起動と動作確認を行った.

実験Ⅱでは,HTMLによる基本的なWebページ作成と,
phpMyAdminを用いたリレーショナルデータベースの構築・検索を行い,
PHPスクリプトと連携させてWebベースの動的なデータ表示や登録が可能であることを確認した.

実験Ⅲでは,Google Maps APIを活用して,
データベースに登録された地理情報をWeb地図上に可視化するインタラクティブなアプリケーションを作成した.
これにより,ユーザは地図上で直感的に情報を取得できるようになり,
Webデータベースの実用性が大きく向上することが示された.

さらに課題では,新たなテーブルの追加やデータ構造の拡張を通じて,
柔軟なデータ管理や表示が可能であることを確認し,実践的な応用力を高めることができた.

今回の実験を通じて,Web開発とデータベース設計の基礎的な知識と,
それらを連携させたシステム構築の手順を体系的に学ぶことができた.

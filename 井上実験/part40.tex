\section{時間応答に基づくパラメータ同定}

\subsection{実験内容}
\subsubsection{パラメータ同定の手順}

\paragraph{Pコントローラ}

\begin{equation}
    v_x(t) = k_{Px}(\theta_x^{\mathrm{ref}}(t) - \theta_x(t)) \tag{4.2}
\end{equation}

によりリンク1の関節角 $\theta_x(t)$ を制御したとき,2軸ロボットの数学モデル(式(2.4))と仮定すると,$\theta_x^{\mathrm{ref}}(s)$ から $\theta_x(s)$ への伝達関数は2次遅れ系

\begin{align}
    T_x(s) = \frac{G_x(s)k_{Px}}{1 + G_x(s)k_{Px}} = \frac{b_x k_{Px}}{s^2 + a_x s + b_x k_{Px}} = \frac{\omega_{nx}^2}{s^2 + 2 \zeta_x \omega_{nx} s + \omega_{nx}^2} \tag{4.3}
\end{align}

\[ 
\left\{
\begin{aligned}
    &\text{固有角周波数:} && \omega_{nx} = \sqrt{b_x k_{Px}} \\
    &\text{減衰係数:} && \zeta_x = \frac{a_x}{2 \omega_{nx}} = \frac{a_x}{2\sqrt{b_x k_{Px}}}
\end{aligned}
\right. \tag{4.4}
\]

となる.

(4.4)式から明らかなように,比例ゲイン $k_{Px}$ を大きくすると $\zeta_x$ は零に近づくため,ステップ応答はオーバーシュートが生じ,振動的になっていく.そこで,ある程度大きな比例ゲイン $k_{Px}$ を設定し,
目標値を

\begin{equation}
    \theta_x^{\mathrm{ref}}(t) = 
    \begin{cases}
        0 & (t < 0) \\
        \theta_{dx} & (t \geq 0)
    \end{cases} \tag{4.5}
\end{equation}

としたステップ応答にオーバシュートを生じさせると,最大ピーク値 $\max \theta_x$ と行き過ぎ時間 $t_{px}$ は以下のようになる.

\begin{equation}
    \left\{
    \begin{aligned}
        \max \theta_x &= \theta_{dx} \left(1 + \exp\left(-\frac{\gamma_x \pi}{\delta_x} \right) \right) \\
        t_{px} &= \frac{\pi}{\delta_x}
    \end{aligned}
    \right. \tag{4.6}
\end{equation}

ただし,

\begin{equation}
    \delta_x = \zeta_x \omega_{nx}, \quad \gamma_x = \omega_{nx} \sqrt{1 - \zeta_x^2} \tag{4.7}
\end{equation}

以上のことから,未知パラメータ $a_x$,$b_x$ の同定手順は以下のようになる.

パラメータ同定の手順
    \begin{enumerate}
        \item ある程度大きい比例ゲイン $k_{Px}$ を設定し,Pコントローラによるリンク1の角度制御を行う.ただし,$\theta_{dx}$ はあまり大きく設定しないようにする.たとえば,$k_{Px} = 20$,$\theta_{dx} = 0.2$ とする.
    
        \item 1) で設定した $k_{Px}$,$\theta_{dx}$ を用いて実機実験を行い,$\theta_x(t)$ をデータ列 $\theta_x^{\mathrm{data}}$ として取得する.
    
        \item 2) で取得したデータ $\theta_x^{\mathrm{data}}$ の最大値 $\max \theta_x^{\mathrm{data}}$ とそのときの時間 $t_{px}^{\mathrm{data}}$ を求める.
    
        \item $\gamma_x$,$\delta_x$ の同定値を以下の式により導出する:
        \[
            \gamma_x^{\mathrm{data}} = -\frac{1}{t_{px}^{\mathrm{data}}} \log_e \left( \frac{\max \theta_x^{\mathrm{data}}}{\theta_{dx}} - 1 \right) \tag{4.8}
        \]
        \[
            \delta_x^{\mathrm{data}} = \frac{\pi}{t_{px}^{\mathrm{data}}} \tag{4.9}
        \]
    
        \item 固有角周波数 $\omega_{nx}$,減衰係数 $\zeta_x$ の同定値を以下で定める:
        \[
            \omega_{nx}^{\mathrm{data}} = \sqrt{ (\gamma_x^{\mathrm{data}})^2 + (\delta_x^{\mathrm{data}})^2 } \tag{4.10}
        \]
        \[
            \zeta_x^{\mathrm{data}} = \frac{\gamma_x^{\mathrm{data}}}{\omega_{nx}^{\mathrm{data}}} \tag{4.11}
        \]
    
        \item $a_x$,$b_x$ の同定値 $a_x^{\mathrm{data}}$,$b_x^{\mathrm{data}}$ を次式により定める:
        \[
            a_x^{\mathrm{data}} = 2 \zeta_x^{\mathrm{data}} \omega_{nx}^{\mathrm{data}} \tag{4.12}
        \]
        \[
            b_x^{\mathrm{data}} = \frac{(\omega_{nx}^{\mathrm{data}})^2}{k_{Px}} \tag{4.13}
        \]
    \end{enumerate}
    
なお,未知パラメータ $a_y$,$b_y$ の同定手順は $a_x$,$b_x$ の同定手順と同様なので,ここでは省略する.

\noindent
\textbf{ステップ 1}:
まず,MATLAB/Simulink を起動し,以下の M ファイル ``\texttt{idarm.m}'' と
実機実験モデル ``\texttt{ex\_P.slx}'' を作成し,フォルダ \texttt{\textyen ID} に保存する.

\begin{figure}[htbp]
    \centering
    \includegraphics[width=0.9\linewidth]{figure/step2_model.pdf}
    \caption{P 制御の Simulink モデル(実機実験モデル ``\texttt{ex\_P.slx}'')}
    \label{fig:simulink_model_p}
\end{figure}

\vspace{1em}
\noindent
\textbf{ステップ 2}:
ステップ 1 の \texttt{\textyen ID} をカレントフォルダとして,実機実験モデル ``\texttt{ex\_P.slx}'' をコンパイルする.

\vspace{1em}
\noindent
\textbf{ステップ 2}:
ステップ1の \texttt{\textyen ID} をカレントフォルダとして,実機実験モデル ``\texttt{ex\_P.slx}'' をコンパイルする.

\vspace{1em}
\noindent
\textbf{ステップ 3}:
モータ軸の角度 $\theta_x(t)$ と $\theta_y(t)$ が $0$~[rad] となるように P 制御する.
実機実験モデル ``\texttt{ex\_P.slx}'' のブロック ``Step'',``Step1'' をダブルクリックし,

\[ 
\begin{array}{ll}
\text{Step}~(\theta_x^{\mathrm{ref}}(t))\!: &
\left\{
\begin{array}{l}
\text{ステップ時間: } 0 \\
\text{初期値: } 0 \\
\text{最終値: } 0 \\
\text{サンプル時間: } 0
\end{array}
\right., \quad
\text{Step1}~(\theta_y^{\mathrm{ref}}(t))\!: 
\left\{
\begin{array}{l}
\text{ステップ時間: } 0 \\
\text{初期値: } 0 \\
\text{最終値: } 0 \\
\text{サンプル時間: } 0
\end{array}
\right.
\end{array}
\]

のように変更して,モータ軸の目標角度 $\theta_x^{\mathrm{ref}}(t)$,$\theta_y^{\mathrm{ref}}(t)$ を設定する.
ターゲットに接続し,実行してアーム角度が $0$ になっていることを確認する.

\vspace{1em}
\noindent
\textbf{ステップ 4}:
実機実験モデル ``\texttt{ex\_P.slx}'' のブロック ``Step'',``Step1'' をダブルクリックし,

\[ 
\begin{array}{ll}
\text{Step}~(\theta_x^{\mathrm{ref}}(t))\!: &
\left\{
\begin{array}{l}
\text{ステップ時間: } 0 \\
\text{初期値: } 0 \\
\text{最終値: } 0.2 \\
\text{サンプル時間: } 0
\end{array}
\right., \quad
\text{Step1}~(\theta_y^{\mathrm{ref}}(t))\!: 
\left\{
\begin{array}{l}
\text{ステップ時間: } 0 \\
\text{初期値: } 0 \\
\text{最終値: } 0.2 \\
\text{サンプル時間: } 0
\end{array}
\right.
\end{array}
\]

のように変更して,モータ軸の目標角度 $\theta_x^{\mathrm{ref}}(t)$,$\theta_y^{\mathrm{ref}}(t)$ を設定する.

\noindent
\textbf{ステップ 5}: MATLAB Command Window で M ファイル ``\texttt{idarm.m}'' を実行する.

\subsection{実験結果}
\begin{tabbing}
\hspace{1cm}\=\kill
\> \texttt{>> idarm} \\
\> \texttt{ax = 40.9262} \hspace{6cm} \texttt{a\_x = 40.9262} \\
\> \texttt{bx = 76.7858} \hspace{6cm} \texttt{b\_x = 76.7858} \\
\> \texttt{ay = 42.1695} \hspace{6cm} \texttt{a\_y = 42.1695} \\
\> \texttt{by = 75.7744} \hspace{6cm} \texttt{b\_y = 75.7744}
\end{tabbing}

図~\ref{fig:step_response2} に示すように,実験データと同定された値を用いたシミュレーションデータが一致していることが確認できた.

\begin{figure}[htbp]
    \centering
    \begin{subfigure}[b]{0.45\linewidth}
        \centering
        \includegraphics[width=\linewidth]{figure/thetax_response2.pdf}
        \caption{$\theta_x(t)$ の応答 ($\theta_x^{\mathrm{ref}}(t) = 0.2$, $\theta_y^{\mathrm{ref}}(t) = 0$)}
    \end{subfigure}
    \hfill
    \begin{subfigure}[b]{0.45\linewidth}
        \centering
        \includegraphics[width=\linewidth]{figure/thetay_response2.pdf}
        \caption{$\theta_y(t)$ の応答 ($\theta_x^{\mathrm{ref}}(t) = 0$, $\theta_y^{\mathrm{ref}}(t) = 0.2$)}
    \end{subfigure}
    \caption{P コントローラを用いたときのステップ応答 ($k_{Px}=20,\ k_{Py}=20$)}
    \label{fig:step_response2}
\end{figure}


\subsection{考察}
実験結果より,未知パラメータ $a_x$,$b_x$,$a_y$,$b_y$ を正確に同定できたことが確認できた.
しかし,定常偏差が残っていることが確認できた.これは,P コントローラの特性に起因するものである.


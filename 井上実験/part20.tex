% !TEX root = main.tex
\section{理論}

本実験では,Web技術とデータベース技術を連携させたWebデータベースを構築する.
これにより,ユーザはWebブラウザを通じて地理情報を直感的に扱うことが可能となる.

\subsection{データベース}

データベースとは,情報を一定の形式で蓄積・管理し,効率的に検索・更新できるようにする仕組みである.
データを一元的に保存することにより,情報の重複や不整合を防ぎ,複数のユーザ間での共有や同時利用が可能となる.

データベースの管理には,データベース管理システム(DBMS: Database Management System)が用いられる.
DBMSは,データの記録・更新・検索・削除といった操作を統一的なインターフェースで提供するソフトウェアであり,
ユーザやアプリケーションプログラムはDBMSを介してデータベースにアクセスする(\cite{mariadb}).

特に広く用いられているのがリレーショナルデータベース管理システム(RDBMS: Relational DBMS)である.
これは,データをテーブル形式で構造化し,各行がレコード,各列がフィールドとして定義される.
RDBMSではSQL(Structured Query Language)と呼ばれる言語を用いて,データの追加・取得・更新・削除といった操作を行う(\cite{mysql}).

本実験では,MySQLと互換性のあるMariaDBを使用し,「施設名」「住所」「緯度・経度」「施設の種別」などの地理情報を扱うテーブルを作成し,Webアプリケーションから動的に操作可能な環境を構築する.

\subsection{Webデータベース}

Webデータベースとは,インターネットを介してWebブラウザからアクセス・操作が可能なデータベースシステムであり,
ユーザがPCやスマートフォンなどの端末を用いて,データの検索や登録,更新などの操作を視覚的かつ直感的に行える環境を提供する.
Webデータベースは,通常,Webサーバ上で動作するアプリケーションとDBMSを組み合わせて構築される.
本実験では,PHPで記述したスクリプトを介して,Webページ上のユーザインターフェースとMariaDBを連携させることにより,
動的に扱えるWebアプリケーションの構築を行う.

\subsection{Google Maps API}
Google Maps APIは,Webサイト上で地図情報を表示・操作できるJavaScriptベースのAPIである.
緯度・経度の情報をもとにマーカーを地図上に表示したり,マーカーをクリックして情報ウィンドウを開くなどのインタラクションが可能となる.

本実験では,Google Maps JavaScript APIを用いて,データベースに保存された地点情報
(施設名,住所,緯度・経度,種別など)を読み込み,マーカーとして地図上に表示する(\cite{googleapi}).
各マーカーには施設の名称や詳細情報が紐づけられており,クリックすることで情報ウィンドウが表示される.

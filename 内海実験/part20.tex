% !TEX root = main.tex
\section{理論}
\subsection{半導体とは}

物質は電子を通す導体(銅,アルミニウムなど)と通さない絶縁体(プラスチック,ゴムなど)と条件を満たすと通す半導体の3つに分かれる.
半導体は14族のSi,Ge,GaAsなどが存在する.現在Siが主に使用されており,理由は大きく4つある.
\begin{enumerate}
    \item 地球で2番目に多い元素のため,資源が豊富である
    \item 不純物を取り除きやすく,高純度化しやすい
    \item 単結晶化と不純物の量を調整して,抵抗率の制御がしやすい
    \item 安定した酸化膜ができ,集積化などの加工がしやすい
\end{enumerate}

純粋なシリコンやゲルマニウムの結晶の性質は絶縁体に近く,電圧をかけても電気はほとんど流れない.
結晶中の電子同士が固く結合していて,自由に動き回れる電子はわずかしかない.
しかし,電子を余計に持っているP(リン)などの不純物をほんの少し加えるだけで,導体のような性質に変化する.
このPのように電子を余計にもった不純物が含まれるものをN型半導体,
逆に電子の少ないホウ素などの不純物が入ったものをP型半導体と呼ぶ.

ダイオードとは電気の流れを一方通行にする半導体素子である.ダイオードは整流(交流を直流に変換する操作),
検波(無線信号から音声信号を取り出す操作)などに使用される.

\subsection{pn接合}

p型半導体とn型半導体を接合した構造をpn接合と呼ぶ.この接合部では,n型半導体中の電子がp型半導体へ拡散し,p型中の正孔と再結合することで空乏層が形成される.同様に,p型の正孔もn型側に拡散し,再結合する.これにより接合部付近には移動キャリアが存在しない空間電荷領域ができ,内部電界が形成される.この内部電界によって拡散電流とドリフト電流が釣り合い,熱平衡状態となる.

pn接合に外部電圧を印加すると,このバリアが変化する.順方向にバイアスをかけると内部電界が弱まり,電流が流れやすくなる.逆方向にバイアスをかけると内部電界が強まり,電流はほとんど流れなくなる.このように,pn接合は整流作用を持つ素子となる.

\subsection{ショットキー接触}

金属と半導体を接触させたとき,両者の仕事関数の違いによりエネルギー障壁が形成されることがある.金属の仕事関数 $\phi_M$ よりも半導体の仕事関数 $\phi_S$ が小さい場合,電子は半導体から金属に移動し,半導体側に空乏層が形成される.これをショットキー接触と呼び,整流特性を示すダイオード動作を実現できる.

このショットキー障壁は金属側からの電子の注入を阻止する働きを持ち,逆バイアスでは電流が流れにくくなる一方,順バイアスでは障壁が下がり,電子が金属に注入され電流が流れる.この特性はpn接合ダイオードと同様の整流性を持つが,キャリアの注入機構が異なるため高速応答性が高い.

\subsection{オーミック接触}

一方,金属の仕事関数 $\phi_M$ よりも半導体の仕事関数 $\phi_S$ が大きい場合,接触界面には障壁が形成されず,電子は容易に移動できる.このような接触はオーミック接触と呼ばれ,I-V特性は線形となり,電流は両方向に自由に流れる.

オーミック接触は素子の電極として重要であり,理想的には接触抵抗が小さくなるように設計される.実際にはドーピング濃度を高くすることで空乏層を極めて薄くし,トンネル効果を利用してオーミック接触を形成する場合が多い.

\subsection{光起電力効果}

pn接合に吸収端波長より短い波長の光を照射すると,電子と正孔の対が生成される.この際,空乏層内で生成されたキャリアは内部電界の影響でそれぞれ異なる方向へドリフトし,接合部を中心に電荷の分離が起こる.

この電荷分離により接合部には電位差が生じ,外部回路に電流を流すことが可能となる.これを光起電力効果(photovoltaic effect)と呼ぶ.太陽電池はこの原理に基づいており,照射された光の強度に応じて起電力が変化する.

なお,光照射が強くなると,空乏層によるエネルギーバリアが減少し,最終的にはバンドがフラットになり,それ以上の電位差は生じなくなる.これにより太陽電池の開放電圧には理論的な上限が存在する.
\subsection{ダイオードの電流-電圧特性と各パラメータの導出方法}

ダイオードの理想的な電流-電圧関係式は以下の式で表される.
\begin{equation}
    I = I_s \left( e^{\frac{q(V - IR_s)}{nkT}} - 1 \right)
\end{equation}
ここで,各記号の意味は以下の通りである:
\begin{itemize}
    \item \( I \):ダイオードを流れる電流 [A]
    \item \( V \):ダイオードに印加される電圧 [V]
    \item \( I_s \):逆方向飽和電流 [A]
    \item \( R_s \):直列抵抗 [\(\Omega\)]
    \item \( n \):理想係数(1〜2)
    \item \( q \):電気素量(\(1.602 \times 10^{-19}\) C)
    \item \( k \):ボルツマン定数(\(1.381 \times 10^{-23}\) J/K)
    \item \( T \):絶対温度 [K](通常 300 K 近傍)
\end{itemize}

\subsubsection*{直列抵抗 \( R_s \) の導出方法}

直列抵抗は,ダイオードの順方向電圧-電流特性の高電圧領域における傾きから求めることができる.この領域では指数項が大きくなり,次のように近似できる:
\begin{equation}
    V \approx IR_s + \frac{nkT}{q} \ln \left( \frac{I}{I_s} \right)
\end{equation}

この式の両辺を \( I \) で微分すると,以下の関係が得られる:
\begin{equation}
    \frac{dV}{dI} = R_s + \frac{nkT}{qI}
\end{equation}

電流 \( I \) が十分大きい領域では,第2項が小さくなるため,おおよそ
\begin{equation}
    \frac{dV}{dI} \approx R_s
\end{equation}
より,I-Vカーブの高電流領域の傾きが \( R_s \) に一致する.したがって,順方向特性の傾きを微分法や近似直線により取得することで \( R_s \) を算出できる.

\subsubsection*{並列抵抗 \( R_p \) の導出方法}

並列抵抗は,逆方向にバイアスをかけた際に流れる漏れ電流による抵抗成分を表す.このときのダイオードはほぼ抵抗として動作するため,以下の式が成り立つ:
\begin{equation}
    R_p = \left| \frac{dV}{dI} \right|_{V < 0}
\end{equation}

逆方向の低電圧領域(線形部分)の I-V グラフの傾きから求めることができる.

\subsubsection*{理想係数 \( n \) の導出方法}

直列抵抗による電圧降下が無視できる場合(低電流領域),式(1)を以下のように近似できる:
\begin{equation}
    I \approx I_s \exp\left( \frac{qV}{nkT} \right)
\end{equation}

両辺の対数をとることで直線関係に変換できる:
\begin{equation}
    \ln I = \ln I_s + \frac{qV}{nkT}
\end{equation}

この式は \( V \) に対して直線であり,傾き \( a \) は以下のようになる:
\begin{equation}
    a = \frac{q}{nkT}
\end{equation}

したがって,傾き \( a \) を求めれば理想係数 \( n \) は次式で求まる:
\begin{equation}
    n = \frac{q}{akT}
\end{equation}

具体的には,順方向 I-V 特性を半対数グラフ(縦軸を \( \log I \),横軸を \( V \))でプロットし,直線部分の傾きを取得することで \( n \) を導出できる.

